%!TEX root = ../main.tex
\preliminar{Agradecimientos}
A mi Madre por su apoyo y ayuda incondicional en todo momento.\par
 A mis Padres por su apoyo incondicional y sus invaluables colaboraciones: Juan P. Rada con la Teoría de Juegos, Raúl E. Encinoza con el Robot. \par
 A mis abuelos: Nona y Remy, Lucía y Enrique; por su ayuda y apoyo. \par
 Al Ing. Douglas Domínguez por su excelente orientación y por presentarme el fascinante mundo de la teoría de juegos. \par
 A FUNDACITE Lara por su excelente labor como fundación. \par
 A la Universidad Fermín Toro por su sobresaliente desempeño en la formación de los estudiantes. \par
 A todos los profesores con quienes tuve el honor de recibir clases, en especial a: Luis Alvarado, Douglas Domínguez, Edecio Freites, Samary Paez y Jorge Rodríguez. \par
 A mis amigos, en especial a Oswaldo “J” Hernández por su tremenda ayuda para llegar a este punto, a Eduardo Molina por su constante apoyo y gran interés en el Equilibrio Nash! \par
 A las chicas que me acompañaron durante esta aventura de 5 años (y un poquito más!), cronológicamente: Olga Pardi, Mardia Ramos, Florangel Rivas, Maritza Pasquarelli y Estefania Alfaro. \par
 Y por supuesto, a todas las personas que colaboraron de una forma u otra en el 
desarrollo de este trabajo de grado. \par
Muchas Gracias. \par
Juan C. Rada. \par