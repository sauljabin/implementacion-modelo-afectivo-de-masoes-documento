%!TEX root = ./main.tex
\includeonly{
chapters/dedicatoria, 
chapters/agradecimientos, 
chapters/introduccion, 
chapters/el-problema, 
chapters/marco-teorico, 
chapters/marco-metodologico, 
chapters/resultados,
chapters/conclusiones,
chapters/recomendaciones,
chapters/anexos
}

\final{false} % En caso de ser final, quita el color de los hipervínculos
\preambulo
\loadglsentries{chapters/glosario.tex}

\begin{document}
		% \titulo{}
		% \autor{}
		% \citarcomo{}  
		% \decanato{Ciencias y Tecnología}
		% \postgrado{Ciencias de la Computación}
		% \ciudad{Barquisimeto}
		% \diadefensa{20}
		% \mesdefensa{Marzo}
		% \annodefensa{2009}
		% \tutor{}
		% \primerjurado{}
		% \segundojurado{}
		% \resumen{}
		% \palabrasclave{}

		% \title{}
		% \abstract{}
		% \keywords{}
		
	\habilitarpendientes	
	\habilitarnotas
	
	\configurar
\begin{preliminares}
	\hacercaratula
	\hacerpresentacion
	\haceraprobacion
	%!TEX root = ../trabajo.tex
\preliminar{Dedicatoria}


	%!TEX root = ../main.tex
\preliminar{Agradecimientos}

	\hacerindice
	\hacerresumen
	\hacerabstract
\end{preliminares}

\begin{contenido}
	%!TEX root = ../main.tex
\introduccion

El desarrollo de un sector de la robótica está dirigido hacia la creación de 
robots cada vez más semejantes al ser humano, tanto en términos corporales como en 
términos de inteligencia. Corporalmente, implica un diseño mecánico para que la 
estructura siempre permanezca en equilibrio. Este equilibrio será estable si la 
estructura mantiene el equilibrio para cualquier sistema de cargas aplicadas, ó 
inestable si se mantiene sólo para un conjunto particular. 

En términos de inteligencia es necesario modelar el raciocinio que tienen los 
seres vivos para reaccionar ante cualquier situación, lo cual siempre conlleva hacia 
una toma de decisiones y, la calidad de éstas, dependerá netamente de la herramienta 
utilizada para crear Inteligencia Artificial. 

 Existen diversas herramientas para tomar decisiones a nivel de inteligencia 
artificial, por ejemplo: las redes neuronales emplean algoritmos de aprendizaje 
iterativo, los sistemas inteligentes utilizan inferencias lógicas, los algoritmos 
genéticos aplican la teoría de la evolución darwiniana, entre otros. 

 La teoría de juegos también permite tomar decisiones. Aunque su principal 
campo de acción se ubica en la economía, ésta describe matemáticamente cualquier 
situación en la que dos o más individuos toman decisiones en búsqueda de un 
resultado que genere bienestar. Dicho bienestar puede ser individual o grupal, y 
dependerá netamente de la estrategia que utilice cada participante.  

 Un concepto fundamental en la teoría de juegos es el Equilibrio Nash. Este 
define una estrategia para cada individuo que participa en el “juego” y, cuando éstas 
son utilizadas, todos los participantes obtendrán el mejor bienestar posible. En otras 
palabras, se habrán tomado las decisiones óptimas. 

 Este trabajo de grado se basa en la construcción de un robot inteligente capaz 
de tomar decisiones óptimas en situaciones finitas de competencia que, a modo de 
ejemplo, están enfocadas al juego Tres en Línea. 

 Este trabajo de grado se estructura como sigue: el Capítulo I presenta la 
descripción del problema, justificación de la investigación, objetivos, alcances y 
limitaciones de la misma. El Capítulo II presenta los antecedentes de la investigación 
junto a las bases teóricas del trabajo, las cuales son divididas en secciones: Mecánica, 
Electrónica y Teoría de Juegos.  

 El Capítulo III describe el marco metodológico donde se encuentra la 
naturaleza del trabajo y las tres fases para el desarrollo de la investigación: 
Diagnóstico, Factibilidad y Diseño. Además, explica brevemente el diseño del 
sistema.  

 El Capítulo IV presenta los resultados obtenidos y se explica detalladamente 
los módulos que integran al sistema; en el Capítulo V se presentan las conclusiones y 
recomendaciones. 
	%!TEX root = ../main.tex
\capitulo{El Problema}

\seccion{Planteamiento del Problema}

 T. Kanade (2004), en entrevista con Popular Mechanics en Español opina que 
\textquotedblleft la mejor definición de robot es aquella que habla de un sistema que registra el 
mundo real exterior para interpretarlo y tomar decisiones útiles e inteligentes que 
tengan un impacto en el mundo real\textquotedblright. A lo que Popular Mechanics en Español (2004) 
considera que \textquotedblleft Es, básicamente, a lo que nos dedicamos los humanos. A decidir.\textquotedblright 	\pendiente{Revisar redaccion}
	
 Muchas situaciones de la vida cotidiana se basa en la toma de decisiones: una 
pareja decide si ir al cine o a un concierto, una empresa decide si entrar ó no a un 
mercado. Estas son decisiones que se toman de acuerdo a intereses basándose en 
instintos, experiencias o incluso en sentido común. En nada difieren de los juegos: en 
ajedrez, dos personas deciden sobre cómo jugar para lograr un jaque mate; en Tres en 
Línea dos personas deciden sobre donde colocar los símbolos X y O para lograr tres 
en línea. 

 La toma de decisiones en robótica requiere de inteligencia artificial, la cual 
posee diversas herramientas para su modelaje. Tales herramientas son: las redes 
neuronales y sus algoritmos de aprendizaje iterativo; los sistemas inteligentes y sus 
inferencias lógicas sobre la base de conocimientos; los algoritmos genéticos y su 
evolución por la teoría darwiniana; entre otras. 

 Aunque con estas herramientas cualquier robot puede llegar a tomar las 
mejores decisiones, el problema radica en extensas situaciones finitas de información 
perfecta con interdependencia estratégica (e.g. Tres en Línea) donde no es fácil 
demostrar que el resultado obtenido es el óptimo. 

 Por ejemplo, se construye un robot capaz de tomar decisiones en el juego Tres 
en Línea y su inteligencia artificial está definida por un sistema inteligente. De esta 
forma, el robot jugará utilizando la experiencia previa y con el tiempo jugará mejor. 
El resultado es empírico, aunque tenga mucha experiencia surge la pregunta: se han 
tomado las mejores decisiones? 

 Utilizando el mismo ejemplo, pero en lugar de utilizar un sistema inteligente 
su inteligencia artificial está definida por una red neuronal. Sin entrar en detalles 
respecto a su diseño e independientemente si son de aprendizaje supervisado o no, el 
robot aprenderá a tomar decisiones. Si toma una decisión errónea, se modifican los 
pesos de las neuronas y, de esta forma, convergerá hacia una buena toma de 
decisiones; pero aunque el error sea mínimo, surge nuevamente la pregunta: se han 
tomado las mejores decisiones? 

 La misma pregunta surge utilizando algoritmos genéticos.  

 Una forma de responder a esta pregunta es analizando todas las posibles 
combinaciones estratégicas del juego para determinar si en realidad se tomaron las 
mejores decisiones, lo cual es poco práctico debido al tamaño del árbol de juego. 
 
Asimismo, esta inquietud no sólo tiene lugar en el juego Tres en Línea, existirá 
siempre que se utilicen estas herramientas para la toma de decisiones en situaciones 
similares. 

  Por el problema expuesto anteriormente y en búsqueda de una herramienta 
que permita tomar decisiones óptimas y justificadas, se plantea el uso de la teoría de 
juegos para crear Inteligencia Artificial y así solventar (de la mejor manera posible) 
cualquier problema de decisión en estas situaciones. Luego, para su aplicación en el 
campo de la robótica, se construirá un robot que utilice esta herramienta para tomar 
decisiones en el juego Tres en Línea.  

\seccion{Objetivos}
	\subseccion{Objetivos Generales}
		Construir un robot inteligente capaz de jugar Tres en Línea. 
		  Por el problema expuesto anteriormente y en búsqueda de una herramienta 
		que permita tomar decisiones óptimas y justificadas, se plantea el uso de la teoría de 
		juegos para crear Inteligencia Artificial y así solventar (de la mejor manera posible) 
		cualquier problema de decisión en estas situaciones. Luego, para su aplicación en el 
		campo de la robótica, se construirá un robot que utilice esta herramienta para tomar 
		decisiones en el juego Tres en Línea.  
		
	\subseccion{Objetivos Específicos}
	\begin{enumeracion}
		\item Diagnosticar la utilidad de la teoría de juegos en problemas para la toma de decisiones en el campo de la robótica. 
		\item Estudiar la factibilidad técnica, económica y operativa del sistema propuesto. 
		\item Construir un robot inteligente capaz de jugar Tres en Línea. 
	\end{enumeracion}

\seccion{Justificación e Importancia}
Es claro que la toma de decisiones es importante para cualquier ser vivo, pero 
más importante aún es tomar una óptima decisión. Este resultado se logra utilizando 
la teoría de juegos que, con fundamentos matemáticos, demuestra que las decisiones 
tomadas son óptimas. 

 Con este trabajo de investigación se pretende introducir la teoría de juegos en 
el campo de la robótica con la construcción de un robot inteligente capaz de tomar 
decisiones óptimas en el juego Tres en Línea. Aunque este juego es sólo un ejemplo 
de situación donde se toman decisiones, se explicará detalladamente la teoría para su 
aplicación a cualquier situación finita de información perfecta.  

 Vale destacar que este trabajo es de carácter innovador dentro de la 
Universidad Fermín Toro, ya que hasta la fecha (Abril de 2005) no se han realizado 
trabajos de grado donde se utiliza la teoría de juegos como herramienta para la toma 
de decisiones a nivel de inteligencia artificial.  

 Una opinión sobre el uso de la teoría de juegos en esta área es presentada por 
J. Roach (2004): \textquotedblleft Los objetivos de estas dos áreas son tan similares que es solamente 
natural, para la teoría de juegos e inteligencia artificial, crear una sinergia para 
formular novedosos enfoques en la solución de diversos problemas\textquotedblright. 
 
Por otro lado, el robot construido será adquirido por la Fundación para la 
Ciencia y Tecnología del Estado Lara (FUNDACITE Lara) para fomentar 
investigaciones en el campo de la robótica. Por este motivo, se creará un robot 
flexible para que pueda ser utilizado en una mayor gama de aplicaciones. 

Este trabajo de grado está orientado al beneficio de la comunidad por las 
siguientes razones 
\begin{enumeracion}
	\item Se explica detalladamente una herramienta que permite tomar decisiones óptimas, 
	la cual podrá ser utilizada para desarrollar desde sistemas hasta robots inteligentes 
	que no sólo tomarán buenas decisiones sino que tomarán las mejores. 
	\item  Se explica el diseño del robot desde un punto de vista mecánico, lo cual permitirá 
	servir de apoyo bibliográfico para las consideraciones que se deben tomar en 
	cuenta al momento de la construcción de cualquier robot. 
	\item El código del software desarrollado en alto y bajo nivel están comentados y bajo 
	el esquema de software libre, lo cual dará una mejor visión sobre cómo crear 
	Inteligencia Artificial utilizando la teoría de juegos además de cómo controlar un 
	robot; de igual manera se podrá disfrutar de los beneficios que ofrece el software 
	libre. 
\end{enumeracion}

 Por lo antes mencionado, este trabajo de grado se ubica en el Polo 2 (Hombre, 
Ciudad y Territorio) que al intersectarlo con el eje correspondiente, diseño y 
mantenimiento de sistemas inteligentes, ubica la línea de investigación Inteligencia 
Artificial y Robótica. 

\seccion{Alcances y Limitaciones}
	\subseccion{Alcances}
	\begin{enumeracion}
		\item Construir un brazo robótico capaz de tomar decisiones óptimas en el juego Tres 
		en Línea. 
		\item Construir un efector final capaz de dibujar la jugada del robot en el campo de 
		juego. 
		\item Conseguir una estrategia óptima para cada jugador.
		\item Interpretar el campo de juego por visión artificial.
		\item Facilitar la interacción humano – robot mediante un tablero de juego.
		\item Integrar el software y el hardware mediante un protocolo de comunicación 
		bidireccional. 
	\end{enumeracion}
	
	\subseccion{Limitaciones}
		\begin{enumeracion}
			\item El robot sólo dibujará su jugada respectiva utilizando el símbolo O. 
			\item Las jugadas realizadas en el campo de juego deberán realizarse dentro de los 
			márgenes establecidos y de un tamaño no menor a 4 cm2.  
			\item Los algoritmos de visión artificial están diseñados para reconocer los símbolos O, 
			cualquier otro símbolo será interpretado como una jugada del adversario. 
			\item  La posición de la cámara digital no podrá alterarse luego de reiniciar el 
			microcontrolador. 
			\item La iluminación del escenario no podrá alterarse. 
			\item El campo de juego deberá permanecer en la posición establecida. 
			\item El robot no evita obstáculos. 
		\end{enumeracion}
	
	%!TEX root = ../trabajo.tex
\capitulo{Marco Teórico}

\seccion{Antecedentes de la Investigación}

\seccion{Bases Teóricas}

\seccion{Bases Legales}

\seccion{Definición de Términos Básicos}

%Saúl Piña - sauljabin@gmail.com
%Si se va a usar el glosario
%\hacerglosario

\seccion{Sistema de Hipótesis}

\seccion{Operacionalización de las Variables}
	%!TEX root = ../trabajo.tex

%%
% Copyright (c) {año} {nombre} <{email}>.
% 
% This program is free software: you can redistribute it and/or modify
% it under the terms of the GNU General Public License as published by
% the Free Software Foundation, either version 3 of the License, or
% (at your option) any later version.
% 
% This program is distributed in the hope that it will be useful,
% but WITHOUT ANY WARRANTY; without even the implied warranty of
% MERCHANTABILITY or FITNESS FOR A PARTICULAR PURPOSE.  See the
% GNU General Public License for more details.
% 
% You should have received a copy of the GNU General Public License
% along with this program.  If not, see <http://www.gnu.org/licenses/>.
%%

\capitulo{Marco Metodológico}

\seccion{Tipo de Investigación}

\seccion{Población y Muestra}

\seccion{Diseño de la Investigación o Procedimiento}

\seccion{Técnicas e Instrumentos de Recolección de Datos}

\seccion{Técnicas de Procesamiento y Análisis de los Datos}

	%!TEX root = ../main.tex
\capitulo{Resultados}
	%!TEX root = ../trabajo.tex
\capitulo{Conclusiones}

	%!TEX root = ../main.tex
\capitulo{Recomendaciones}
\end{contenido}

\hacerbibliografia{main}

\begin{anexos}
	\haceranexos
	%!TEX root = ../main.tex
\anexo{Curriculum Vitae}


\end{anexos}
	
	
\end{document}	
