%!TEX root = ./main.tex
\agregartermino{algoritmo-evolutivo}{
	name={Algoritmos Evolutivos}, 
	text={algoritmo evolutivo},
	description={son algoritmos dentro de la Computación Evolutiva que se inspiran en los mecanismos biológicos de selección, cruce, reproducción, y mutación para conseguir soluciones óptimas o sub-óptimas a problemas de optimización. Están basados en la evolución de poblaciones donde cada individuo representa una solución y existe una \gls{funcion-aptitud} que determina la calidad de la solución y guía el proceso evolutivo}, 
	plural={algoritmos evolutivos}
}

\agregartermino{algoritmo-genetico}{
	name={Algoritmos Genéticos}, 
	text={algoritmo genético},
	description={es un \gls{algoritmo-evolutivo} cuyas soluciones se denominan cromosomas y son representados generalmente en cadenas binarias. Por ser un \gls{algoritmo-evolutivo}, se inspira en los mecanismos biológicos de selección, cruce, reproducción, y mutación}, 
	plural={algoritmos genéticos}
}

\agregartermino{bullet-physics-engine}{
	name={Bullet Physics Engine}, 
	text={\texttt{Bullet Physics Engine}},
	description={es un \gls{motor-fisica} desarrollado por Erwin Coumanns. Su  código fuente es gratuito y se encuentra disponible bajo la licencia zlib en \url{http://www.bulletphysics.com/}}, 
	plural={Bullet Physics Engine}
}

\agregartermino{cuaternion}{
	name={Cuaternión}, 
	text={cuaternión},
	description={es un sistema numérico no conmutativo que extiende a los números complejos. Sirve para representar orientaciones y rotaciones de objetos en tres dimensiones. Se define como $q = w + x\mathbf{i} + y\mathbf{j} + z\mathbf{k}$ (\citar{wiki:quaternion})},
	plural={cuaterniones}
}

\agregartermino{escalamiento-likert}{
	name={Escalamiento Likert}, 
	text={escalamiento Likert},
	description={consiste en presentar un conjunto de ítems en forma de afirmaciones o juicios sobre los cuales se pide la reacción de los sujetos (\citar{Sampieri03})}, 
	plural={escalamiento Likert}
}

\agregartermino{equilibrio-estatico-cuerpos-rigidos}{
	name={Equilibrio Estático de Cuerpos Rígidos}, 
	text={equilibrio estático de cuerpos rígidos},
	description={un cuerpo rígido está en equilibrio estático cuando se cumple que la suma de fuerzas en todas las partículas del sistema es cero, y también lo es la suma de momentos en todas las partículas del sistema (\citar{wiki:mechanical-equilibrium})}, 
	plural={equilibrio estático de cuerpos rígidos}
}

\agregartermino{funcion-aptitud}{
	name={Función de Aptitud}, 
	text={función de aptitud},
	description={es una función que mide de manera objetiva y determinista la calidad de una solución}, 
	plural={funciones de aptitud}
}


\agregartermino{gnu-gpl}{
	name={GNU General Public License}, 
	text={GNU General Public License},
	description={el software bajo esta licencia otorga las siguientes libertades a los usuarios: 
	\begin{enumeracionenparrafo}
		\item libertad para usar el software con cualquier propósito,
		\item libertad para cambiar el software a fin de satisfacer sus necesidades,
		\item libertad para compartirlo con amigos y vecinos,
		\item libertad para compartir los cambios hechos por los usuarios
	\end{enumeracionenparrafo}(\citar{Smith07})},
	plural={GNU General Public License}
}


\agregartermino{grado-libertad}{
	name={Grados de Libertad}, 
	text={grado de libertad},
	description={son el conjunto de desplazamientos y rotaciones independientes que alteran la posición y orientación de un cuerpo o sistema. Un cuerpo rígido de $n$ dimensiones tiene un total de $(n + 1) / 2$ \glspl{grado-libertad}: $n$ de traslación y $n*(n-1)/2$ de rotación (\citar{wiki:dof})}, 
	plural={grados de libertad}
}

\agregartermino{inteligencia-enjambres}{
	name={Inteligencia de Enjambres}, 
	text={inteligencia de enjambres},
	description={según \citar{wiki:swarm-intelligence}, es una clase de inteligencia artificial basada en el comportamiento cooperativo de sistemas descentralizados y autoorganizados}, 
	plural={inteligencia de enjambres}
}

\agregartermino{momento}{
	name={Momento}, 
	text={momento},
	description={también conocido como \emph{torque}, es la tendencia de una fuerza a rotar un objeto sobre algún eje}, 
	plural={momentos}
}

\agregartermino{motor-fisica}{
	name={Motor de Física}, 
	text={motor de física},
	description={es un programa de computación que realiza simulaciones utilizando las leyes newtonianas de la física, tomando en cuenta variables como masa, velocidad, fricción, y resistencia al viento (\citar{wiki:physics-engine}) }, 
	plural={motores de física}
}

\agregartermino{neuroevolucion}{
	name={Neuroevolución}, 
	text={neuroevolución},
	description={consiste en entrenar una \gls{red-neuronal-artificial} utilizando \glspl{algoritmo-evolutivo}. Se pueden hacer evolucionar los pesos de las conexiones, la arquitectura, o bien la regla de aprendizaje (\citar{Yao99})}, 
	plural={neuroevolución}
}


% \agregartermino{neurona-mccullogh-pitts}{
% 	name={Neuronas McCullogh-Pitts}, 
% 	text={neurona McCullogh-Pitts},
% 	description={son neuronas caracterizadas por entradas y salidas binarias, no existen pesos en las conexiones, y la función de activación siempre es la función escalón. Están compuestas por un conjunto de entradas excitadoras, un conjunto de entradas inhibidoras, un umbral, la función de activación escalón, y una neurona de salida}, 
% 	plural={neuronas McCullogh-Pitts}
% }
\agregartermino{neurona-mccullogh-pitts}{
	name={Neuronas McCullogh-Pitts}, 
	text={neurona McCullogh-Pitts},
	description={fue el primer modelo de neurona artificial. Esta neurona artificial utiliza entradas y salidas binarias, algunas restricciones sobre los posibles valores de los pesos, y un umbral flexible. La función de activación es la función escalón Heaviside (\citar{wiki:artificial-neuron})}, 
	plural={neuronas McCullogh-Pitts}
}


\agregartermino{optimizacion-enjambre-particulas}{
	name={Optimización por Enjambre de Partículas}, 
	text={optimización por enjambre de partículas},
	description={es un algoritmo de la \gls{inteligencia-enjambres} fundamentado en la inteligencia colectiva para conseguir soluciones a un problema de optimización en un espacio de búsqueda. Asimismo, sirve para modelar y predecir comportamiento colectivo cuando se está en presencia de objetivos (\citar{wiki:pso})}, 
	plural={optimizaciones por enjambres de partículas}
}

\agregartermino{propioceptivo}{
	name={Propioceptivo}, 
	text={propioceptivo},
	description={según \citar{AppleDictionary05}, relativo a los estímulos que son producidos y percibidos en un organismo, especialmente aquellos relacionados con la posición y movimiento del cuerpo}, 
	plural={propioceptivos}
}

\agregartermino{red-neuronal-artificial}{
	name={Red Neuronal Artificial}, 
	text={red neuronal artificial},
	description={según \citar{wiki:ann}, es un modelo matemático o computacional basado en las redes neuronales biológicas. Consiste en un grupo de neuronas artificiales interconectadas que procesan información utilizando un enfoque conexionista. En la mayoría de casos es un sistema adaptativo que cambia su estructura basándose en la información interna o externa que se propaga a través de la red durante la fase de aprendizaje}, 
	plural={redes neuronales artificiales}
}

\agregartermino{red-neuronal-recurrente}{
	name={Redes Neuronales Recurrentes}, 
	text={red neuronal recurrente},
	description={es una clase de \gls{red-neuronal-artificial} que utiliza una capa adicional denominada \emph{capa de contexto} o \emph{capa de estado}, la cual permite que la red tenga memoria de corto plazo haciendo que la propagación anterior afecten al comportamiento en curso (\citar{Boden01})}, 
	plural={redes neuronales recurrentes}
}

\agregartermino{sistema-clasificador-patrones}{
	name={Sistema Clasificador de Patrones}, 
	text={sistema clasificador de patrones},
	description={es un sistema de aprendizaje de máquina que consiste en una población de reglas binarias que se hacen evolucionar utilizando \glspl{algoritmo-genetico}, donde la \gls{funcion-aptitud} de cada regla es determinada utilizando técnicas de aprendizaje por reforzamiento (\citar{wiki:learning-classifier-system})},
	plural={sistemas clasificadores de patrones}
}

\agregartermino{sintesis-espectral}{
	name={Síntesis Espectral}, 
	text={síntesis espectral},
	description={es una implementación de la transformada discreta inversa de Fourier, la cual toma una función a partir de un dominio de frecuencias en el dominio espacial (\citar{Ebert02})}, 
	plural={síntesis espectrales}
}


% \agregartermino{}{
% 	name={}, 
% 	text={},
% 	description={}, 
% 	plural={}
% }

% \agregartermino{}{
% 	name={}, 
% 	text={},
% 	description={}, 
% 	plural={}
% }

% \agregartermino{}{
% 	name={}, 
% 	text={},
% 	description={}, 
% 	plural={}
% }

% \agregartermino{}{
% 	name={}, 
% 	text={},
% 	description={}, 
% 	plural={}
% }

% \agregartermino{}{
% 	name={}, 
% 	text={},
% 	description={}, 
% 	plural={}
% }

% \agregartermino{}{
% 	name={}, 
% 	text={},
% 	description={}, 
% 	plural={}
% }

% \agregartermino{}{
% 	name={}, 
% 	text={},
% 	description={}, 
% 	plural={}
% }

% \agregartermino{}{
% 	name={}, 
% 	text={},
% 	description={}, 
% 	plural={}
% }

% \agregartermino{}{
% 	name={}, 
% 	text={},
% 	description={}, 
% 	plural={}
% }

% \agregartermino{}{
% 	name={}, 
% 	text={},
% 	description={}, 
% 	plural={}
% }

% \agregartermino{}{
% 	name={}, 
% 	text={},
% 	description={}, 
% 	plural={}
% }

% \agregartermino{}{
% 	name={}, 
% 	text={},
% 	description={}, 
% 	plural={}
% }

% \agregartermino{}{
% 	name={}, 
% 	text={},
% 	description={}, 
% 	plural={}
% }

% \agregartermino{}{
% 	name={}, 
% 	text={},
% 	description={}, 
% 	plural={}
% }

% \agregartermino{}{
% 	name={}, 
% 	text={},
% 	description={}, 
% 	plural={}
% }

% \agregartermino{}{
% 	name={}, 
% 	text={},
% 	description={}, 
% 	plural={}
% }

% \agregartermino{}{
% 	name={}, 
% 	text={},
% 	description={}, 
% 	plural={}
% }

% \agregartermino{}{
% 	name={}, 
% 	text={},
% 	description={}, 
% 	plural={}
% }

% \agregartermino{}{
% 	name={}, 
% 	text={},
% 	description={}, 
% 	plural={}
% }

% \agregartermino{}{
% 	name={}, 
% 	text={},
% 	description={}, 
% 	plural={}
% }

% \agregartermino{}{
% 	name={}, 
% 	text={},
% 	description={}, 
% 	plural={}
% }

% \agregartermino{}{
% 	name={}, 
% 	text={},
% 	description={}, 
% 	plural={}
% }

% \agregartermino{}{
% 	name={}, 
% 	text={},
% 	description={}, 
% 	plural={}
% }

% \agregartermino{}{
% 	name={}, 
% 	text={},
% 	description={}, 
% 	plural={}
% }

