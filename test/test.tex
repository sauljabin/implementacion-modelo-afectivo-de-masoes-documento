\documentclass{uclamsc}

% Archivo para pruebas

\bibliografia{test}

\begin{document}

\begin{preliminares}
	\hacerindices
\end{preliminares}

\begin{contenido}

Este documento se generó para probar la clase \comillas{uclamsc} v\uclamscversion.

Prueba de \refalgoritmo{alg}

\begin{algoritmo}[titulo=Algoritmo, indice=Indice del algoritmo, etiqueta=alg]
Inicialización\;
\For{$k \leftarrow 1$ \KwTo $m$}{
    Mostrar $k$
}
\end{algoritmo}

Prueba de \refilustracion{ilustracion}

\begin{ilustracion}[fuente=Prueba de fuente\cite{wikibib}, etiqueta=ilustracion, titulo=Prueba de ilustracion, indice=Indice de prueba para ilustración]
	\includegraphics[width=5cm]{escudo-ucla.jpg}
\end{ilustracion}

Prueba de \refgrafico{grafico}

\begin{grafico}[fuente=Una fuente, etiqueta=grafico, titulo=Prueba de grafico, indice=Indice de prueba para gŕafico]
	\includegraphics[width=5cm]{escudo-ucla.jpg}
\end{grafico}

Prueba de \reflistado{listado}

\begin{listado}[titulo=Listado prueba, indice= Indice de listado prueba, etiqueta=listado, fuente=prueba]{[LaTeX]TeX}
\begin{ilustracion}
  \includegraphics{ruta_archivo}
\end{ilustracion}
\end{listado}

Prueba de \refcuadro{ejemplocua}

\begin{cuadro}[titulo={Piedra, Papel ó Tijeras – Forma Normal}, indice=Este es el indice personalizado,etiqueta=ejemplocua]{lccc}
	\toprule
	Jugares I/II & Piedra & Papel & Tijeras\\
	\midrule
	Piedra   & (0,0) & (-1,1) & (1,-1)\\
	Papel   & (1,-1) & (0,0) & (-1,1)\\
	Tijeras   & (-1,1) & (1,-1) & (0,0)\\
	\bottomrule
	\fuentecuadro{4}{Esta es la fuente}
\end{cuadro}

\end{contenido}

\hacerbibliografia

\end{document}
