\capitulo{Ambientes Indexados y Especiales}

Se puede hacer referencia a los ambientes con etiquetas mediante el uso del comando \co{ref}\pa[etiqueta].

\seccion{Ilustración}

\am{ilustracion}{ permite mostrar una ilustración}{[parametros]}{\co{includegraphics}\pa[ruta\_archivo]}
Se puede hacer referencia a una ilustracion usando el comando \co{refilustracion}\pa.

\begin{cuadro}[titulo=Lista de parametros para el ambiente \comillas{ilustración}]{lll}
	\toprule
	Parametro & Descripción & Ejemplo\\
	\midrule
	titulo   & Asigna un título a la ilustración & titulo={Este es el título} \\
	etiqueta & Valor que pemite hacer referencia a la ilustración & etiqueta=ilustracion1 \\
	fuente & Permite agregar la fuente & fuente=\co{cite}\pa[bib] \\
	indice & Permite personalizar el índice & indice={Índice personalizado}\\
	sinleyenda & Evita que se muestre el título & sinleyenda\\
	sinindice & Evita que se muestre en el índice & sinindice\\
	\bottomrule
\end{cuadro}

\seccion{Gráfico}

\am{grafico}{ para mostrar un gráfico}{[parametros]}{\co{includegraphics}\pa[ruta\_archivo]}

Se puede hacer referencia a un grafico usando el comando \co{refgrafico}.

\begin{cuadro}[titulo=Lista de parametros para el ambiente \comillas{grafico}]{lll}
	\toprule
	Parametro & Descripción & Ejemplo\\
	\midrule
	titulo   & Asigna un título al gráfico & titulo={Este es el título} \\
	etiqueta   & Valor que pemite hacer referencia al gráfico & etiqueta=grafico1 \\
	fuente & Permite agregar la fuente & fuente=\co{cite}\pa[bib] \\
	indice & Permite personalizar el índice & indice={Índice personalizado}\\
	sinleyenda & Evita que se muestre el título & sinleyenda\\
	sinindice & Evita que se muestre en el índice & sinindice\\
	\bottomrule
\end{cuadro}

\seccion{Cuadro}

\am{cuadro}{ para mostrar una tabla y agregarla al índice de cuadros}{[parametros]\pa[columnas]}{\ldots}

Utilizar el comando \co{fuentecuadro}\pa[\# columnas en el cuadro]\pa[fuente]\ dentro del ambiente para agregar la fuente de donde se tomó el cuadro.

Véase el uso del ambiente \texttt{\textbf{tabular}} o del paquete \textbf{ctable}.

Se puede hacer referencia a un cuadro usando el comando \co{refcuadro}.

Referencia rápida en \url{http://en.wikibooks.org/wiki/LaTeX/Tables}.

\begin{cuadro}[titulo=Lista de parametros para el ambiente \comillas{cuadro}]{lll}
	\toprule
	Parametro & Descripción & Ejemplo\\
	\midrule
	titulo   & Asigna un título al cuadre & titulo={Este es el título} \\
	etiqueta   & Valor que pemite hacer referencia al cuadro & etiqueta=cuadro1 \\
	indice & Permite personalizar el índice & indice={Índice personalizado}\\
	sinleyenda & Evita que se muestre el título  & sinleyenda\\
	sinindice & Evita que se muestre en el índice & sinindice\\
	columnas & Parametro \textbf{obligatorio}, configura las columnas & lcc\\
	\bottomrule
\end{cuadro}


\seccion{Cita textual}

\am{citatextual}{ Prepara el contexto para una cita textual de más de 40 palabras}
{}{...Esto es una cita textual... (p. 199)}


\seccion{Ecuación}

\am{ecuacion}{ para insertar una ecuación con etiqueta para hacerle referencia}
{\pa[etiqueta]}{y = x + 1}

\espaciodoble

\am{ecuaciones}{ para insertar una lista de ecuaciones numeradas}
{}{y = x + 1 \textbackslash\textbackslash\\ z = x + 2}

\seccion{Listado}

\am{listado}{ para insertar un listado, usado principalmente para códigos y algoritmos, en el parámetro puede indicarse un lenguaje ya soportado, uno definido por el usuario o nada}
{[parametros]\pa[lenguaje]}{\ldots}

Se puede hacer referencia a un grafico usando el comando \co{reflistado}.

Más información en \url{http://en.wikibooks.org/wiki/LaTeX/Source_Code_Listings}.

Utilizar el comando \co{letralistados} para restablecer la letra por defecto, utilizar con un parámetro opcional (entre \texttt{[]}) para cambiar el estilo de la letra.

Utilizar el comando \co{lstdefinelanguage}\pa\pa\ y la clave \texttt{morekeywords} del paquete \textbf{listings} para definir un lenguaje personalizado. Véase la documentación.

Utilizar el comando \co{definirliterales}\pa\ para definir literales de reemplazo. Los parámetros tienen la forma \texttt{\pa[match]\pa[\pa[reemplazo]]largo} y van separados sólo por espacios. Véase la documentación. Ejemplo:

\definirliterales{{~}{{}}0}

\begin{listado}[sinleyenda, sinindice]{[LaTeX]TeX}
\definirliterales{
	{:=}{{$\leftarrow$}}2
	{<<}{{$\langle$}}1 {>>}{{$\rangle$}}1
}
\begin{listado~}[sinleyenda, sinindice]{}
 x := <<valor>>
\end{listado~}
\end{listado}

\vspace{-16pt}\noindent
resulta en:
\vspace{-12pt}

\definirliterales{{:=}{{$\leftarrow$}}2 {<<}{{$\langle$}}1 {>>}{{$\rangle$}}1}

\begin{listado}[sinleyenda, sinindice]{}
 x := <<valor>>
\end{listado}

\begin{cuadro}[titulo=Lista de parametros para el ambiente \comillas{listado}]{lll}
	\toprule
	Parametro & Descripción & Ejemplo\\
	\midrule
	titulo   & Asigna un título al listado & titulo={Este es el título} \\
	etiqueta   & Valor que pemite hacer referencia al listado & etiqueta=listado1 \\
	fuente & Permite agregar la fuente & fuente=\co{cite}\pa[bib] \\
	indice & Permite personalizar el índice & indice={Índice personalizado}\\
	sinleyenda & Evita que se muestre el título  & sinleyenda\\
	sinindice & Evita que se muestre en el índice & sinindice\\
	lenguaje & Parametro \textbf{obligatorio}, configura lel lenguaje & [LaTeX]TeX\\
	\bottomrule
\end{cuadro}


\seccion{Algoritmo}

\am{algoritmo}{ para escribir algoritmos en pseudocodigo. Los parámetros opcionales lo agregan al índice de algoritmos}{[parametros]}{\dots}

Se puede hacer referencia a un algoritmo usando el comando \co{refalgoritmo}.

Este ambiente usa el paquete \textbf{algorithm2e} \url{http://en.wikibooks.org/wiki/LaTeX/Algorithms#Typesetting_using_the_algorithm2e_package}.

Se puede ver el codigo fuente del \refalgoritmo{alg:ejemplo}

\begin{algoritmo}[etiqueta=alg:ejemplo, titulo=Ejemple de algoritmo con fuente, fuente=\cite{wikialg}]
 \KwData{this text}
 \KwResult{how to write algorithm with \LaTeX2e }
 initialization\;
 \While{not at end of this document}{
  read current\;
  \eIf{understand}{
   go to next section\;
   current section becomes this one\;
   }{
   go back to the beginning of current section\;
  }
 }
\end{algoritmo}

\begin{cuadro}[titulo=Lista de parametros para el ambiente \comillas{algoritmo}]{lll}
	\toprule
	Parametro & Descripción & Ejemplo\\
	\midrule
	titulo   & Asigna un título al algoritmo & titulo={Este es el título} \\
	etiqueta   & Valor que pemite hacer referencia al algoritmo & etiqueta=algoritmo1 \\
	fuente & Permite agregar la fuente & fuente=\co{cite}\pa[bib] \\
	indice & Permite personalizar el índice & indice={Índice personalizado}\\
	sinleyenda & Evita que se muestre el título  & sinleyenda\\
	sinindice & Evita que se muestre en el índice & sinindice\\
	\bottomrule
\end{cuadro}
