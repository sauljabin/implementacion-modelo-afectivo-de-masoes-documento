%!TEX root = ../trabajo.tex 

\capitulo{Glosario}

\co{hacerglosario}\ ajusta y titula ``Definición de Términos Básicos'' la sección de términos del glosario insertada por el comando \co{printglossary}. Véase uso del paquete \textbf{glossaries} (\url{http://www.ctan.org/tex-archive/macros/latex/contrib/glossaries}).

Referencia rápida en \url{http://en.wikibooks.org/wiki/LaTeX/Glossary}.

\co{hacerglosarioconacronimos}, igual que \co{hacerglosario}\ pero titulando la sección ``Glosario de Acrónimos y Términos'', más apropiado para cuando se definen acrónimos en el glosario.

Para agregar un término al glosario se debe usar el comando \co{agregartermino}, 
para referirse al término usar el comando \co{gls}. Los términos deben ser agregados en el archivo \textbf{capitulos/glosario}, y debe incluirse en el cuerpo
del documento con el comando \co{input}\pa[capitulos/glosario].

A continuación un ejemplo de como agregar un término.

\begin{listado}[titulo=Ejemplo de término para el glosario]{[LaTeX]TeX}
\agregartermino{latex}{
  name={Latex}, 
  text={Latex},
  description={
    Es un sistema de composición de textos, 
    orientado especialmente a la creación de libros,
    documentos científicos y técnicos que contengan 
    fórmulas matemáticas
  }
}
\end{listado}

A continuación se muestra el resultado del comando \co{hacerglosario}. Se invoca el termino \gls{latex}.

\hacerglosario