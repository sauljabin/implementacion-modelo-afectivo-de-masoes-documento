%%
% Copyright (c) 2017 Saúl Piña <sauljabin@gmail.com>.
%
% This program is free software: you can redistribute it and/or modify
% it under the terms of the GNU General Public License as published by
% the Free Software Foundation, either version 3 of the License, or
% (at your option) any later version.
%
% This program is distributed in the hope that it will be useful,
% but WITHOUT ANY WARRANTY; without even the implied warranty of
% MERCHANTABILITY or FITNESS FOR A PARTICULAR PURPOSE.  See the
% GNU General Public License for more details.
%
% You should have received a copy of the GNU General Public License
% along with this program.  If not, see <http://www.gnu.org/licenses/>.
%%

\preliminar{Resumen Curricular de la Tutora}

\noindent\textbf{DATOS PERSONALES}

\espaciodoble

\textbf{Nombres y Apellidos:} Niriaska Del Carmen Perozo Guédez.

\textbf{Fecha de Nacimiento:}  22/08/1973.

\textbf{Lugar Nacimiento:} Barquisimeto (Edo. Lara - Venezuela).

\textbf{Nacionalidad:} Venezolana.

\textbf{Estado Civil:} Casada.

\textbf{Cédula de Identidad:} 11.593.374.

\textbf{Teléfonos:} 0251-2591720 (Oficina).

\textbf{Móvil:} 0424-5159619.

\textbf{Correo Electrónico:} nperozo@ucla.edu.ve.

\espaciotriple

\noindent\textbf{ESTUDIOS REALIZADOS}

\begin{enumerate}
  \item Secundaria
  \begin{enumerate}[label*=\arabic*.]
    \item \textbf{Institución:} \comillas{Mario Briceño Iragorry}.\\
          \textbf{Año:} 1989-1990.\\
          \textbf{Título Obtenido:} Bachiller en Ciencias.
  \end{enumerate}

  \item Educación Superior
  \begin{enumerate}[label*=\arabic*.]
    \item \textbf{Institución:} \comillas{Universidad Centro Occidental Lisandro Alvarado}.\\
          \textbf{Año:} 1990-1997.\\
          \textbf{Título Obtenido:} Ingeniero en Informática con la Mención Cum Laude.
  \end{enumerate}

  \item Postgrado
  \begin{enumerate}[label*=\arabic*.]
    \item \textbf{Institución:} \comillas{Universidad Paul Sabatier}, Toulouse-Francia.\\
          \textbf{Año:} 2007 – 2011.\\
          \textbf{Título Obtenido:} Doctor en Neurociencias, Cognición y Comportamiento Colectivo.
    \item \textbf{Institución:} \comillas{Universidad de los Andes}, Mérida-Venezuela.\\
          \textbf{Año:} 2006 – 2011.\\
          \textbf{Título Obtenido:} Doctor en Ciencias Aplicadas.
    \item \textbf{Institución:} \comillas{Universidad de los Andes}, Mérida-Venezuela.\\
          \textbf{Año:} 2001 – 2003.\\
          \textbf{Título Obtenido:} Magíster scientiae en Computación.
  \end{enumerate}
\end{enumerate}

\espaciotriple

\noindent\textbf{PROYECTOS DE INVESTIGACIÓN}

\begin{enumerate}

\item Aplicación del Algoritmo de Optimización por Colonia de
Hormigas en la Comparación De Ontologías. Noviembre 2015-Actualmente. Proyecto
que se está desarrollando en la Unidad de Inteligencia Artificial del DCyT.
Código del C.D.C.H.T. – UCLA: Por Asignar.

\item Esquemas de Aprendizaje Ontológico para Sistemas Auto-Organizados y
Emergentes. Junio 2014-Actualmente. Proyecto que se está desarrollando en la
Unidad de Inteligencia Artificial del DCyT. Código del C.D.C.H.T. – UCLA:
001-DCT-2015.

\item Construcción Cooperativa de Arquitecturas 3D Mediante la Optimización por
Enjambre de Partículas. Diciembre 2012 - 2014. Proyecto desarrollado en la
Unidad de Inteligencia Artificial del DCyT. Código del C.D.C.H.T. – UCLA:
004-RCT-2013.

\item Estudio Experimental y Modelización de los Desplazamientos Colectivos de
Grupos de Peatones. Febrero 2007 – Septiembre 2011. Proyecto que se desarrolló
en el Centre de Recherches sur la Cognition Animale (CRCA) de la Universidad
Paul Sabatier en el marco de mis estudios doctorales bajo la tutoría del Dr. Guy
Theraulaz. Cabe señalar, que estos estudios fueron financiados por el Programa
Alban.

\item Esquemas de Coordinación Emergente para Plataformas Distribuidas.
Septiembre 2008-2011. Proyecto que se desarrolló en el Centro de Estudios en
Microelectrónica y Sistemas Distribuidos (CEMISID) de la Universidad de los
Andes en el marco de mis estudios doctorales bajo la tutoría del Dr. José
Aguilar y el Dr. Oswaldo Terán. Código del C.D.C.H.T. – UCLA: 004-RCT-2008.

\item Diseño de un Modelo autorganizativo para un Sistema Manejador de
Comunidades en un Sistema Operativo Web, utilizando Inteligencia Artificial.
Febrero 2005 – Febrero 2008. Proyecto desarrollado en la Unidad de Investigación
en Inteligencia Artificial de la Universidad Centroccidental \comillas{Lisandro
Alvarado}. Código del C.D.C.H.T. – UCLA: 005-RCT-2005.

\end{enumerate}

\espaciotriple

\noindent\textbf{ARTICULOS PUBLICADOS EN REVISTAS}

\begin{enumerate}
\item \comillas{Un Algoritmo Híbrido para la Construcción Cooperativa de Arquitecturas 3D}. Perozo, N., Zapata, H., Angulo, W, Contreras, J. Enviado para publicación a la Revista Técnica de la Facultad de Ingeniería Universidad del Zulia, Venezuela, 2016.
\item \comillas{Self-Organization and Emergence Phenomena in Wikipedia and Free Software Development Using MASOES}. Perozo, N., Aguilar, J., Terán, O., Molina, H. Publicaciones en Ciencias y Tecnología, Decanato de Ciencias y Tecnología, Universidad Centroccidental \comillas{Lisandro Alvarado}, Vol. 7, Número 1, 2013.
\item \comillas{A Verification Method for MASOES}. Perozo, N., Aguilar, J., Terán, O., Molina, H. IEEE Transactions on Systems, Man, and Cybernetics, Part B: Cybernetics, Vol. 43, Issue 1, p.p. 64-76, 2013.
\item \comillas{Un Modelo Afectivo para una Arquitectura Multiagente para Sistemas Emergentes y AutoOrganizados (MASOES)}. Perozo, N., Aguilar, J., Terán, O. Revista Técnica de la Facultad de Ingeniería Universidad del Zulia, vol. 35, Nro. 1, p.p. 1-11, Venezuela, 2012.
\item \comillas{The Walking Behaviour of Pedestrian Social Groups and its Impact on Crowd Dynamics}. Moussaïd, M., Perozo, N., Garnier, S., Helbing, D., Theraulaz, G. Publicado en PLoS ONE 5(4): e10047. doi: 10.1371/journal.pone. 0010047, 2010.
\item \comillas{Proposal for a Multiagent Architecture for Self-Organizing Systems (MA-SOS)}. Autores: Perozo, N., Aguilar, J., Terán, O. Publicado en Lecture Notes in Computer Science Vol. 5075, pp. 434-439, 2008.
\item \comillas{Modelo de un Sistema Manejador de Comunidades Autorganizativo para un Sistema Operativo Web Multiagente}. Perozo, N., Romero, D. Publicado en Publicaciones en Ciencias y Tecnología. Decanato de Ciencias y Tecnología, Universidad Centroccidental \comillas{Lisandro Alvarado}, Vol. II, Nº 2 Diciembre 2008.
\item \comillas{Design of a Community Manager System for a Multiagent Web Operating System}. Perozo, N., Aguilar, J. Publicado en Publicaciones en Ciencias y Tecnología. Decanato de Ciencias y Tecnología, Universidad Centroccidental \comillas{Lisandro Alvarado}, Vol. II, Nº 1 Junio 2008.
\item \comillas{Definition of a Verification Method for the MASINA Methodology}. Autores: J. Aguilar, N. Perozo, J. Vizcarrondo. International Journal of Information Technology Vol. 12, No.3, 2006.
\item \comillas{Architecture of a Web Operating System based on Multiagent Systems}. Autores: J. Aguilar, N. Perozo, E. Ferrer, J. Vizcarrondo, Publicado en Lecture Notes in Artificial Intelligence, Springer-Verlag, Vol. 3681, pp. 700-706, 2005.
\item \comillas{Arquitectura de un Sistema Operativo Web Basado en Sistemas Multiagente}. Autores: J. Aguilar, N. Perozo, E. Ferrer, J. Vizcarrondo. Publicado en la Revista Colombiana de Computación, Vol. 5, Nº 2, pp 7-29, 2004.
\item \comillas{Arquitectura de un Sistema Operativo Web}. Autores: J. Aguilar, N. Perozo, E. Ferrer, J. Vizcarrondo. Publicado en la Revista Gerencia, Tecnología, Informática (electronic journal: www.cidlisuis.org/aedo), Instituto Tecnológico Iberoamericano de Colombia, Vol. 1, No 2, 13 páginas, Julio 2003.
\item \comillas{Sparse Distributed Memory with Adaptive Threshold}. Autores: J. Aguilar, N. Perozo. Publicado en el libro: Soft Computing Systems: Design, Management and Applications. (Ed. A. Abraham, M. Keepen, J. Ruiz). IOS Press Holland, Vol. 87, pp. 426-432, Diciembre 2002.
\end{enumerate}

\espaciotriple

\noindent\textbf{PONENCIAS, CONFERENCIAS O TUTORIALES PRESENTADOS EN
EVENTOS CIENTÍFICOS NACIONALES}

\begin{enumerate}
\item \comillas{Simulación del Auto-Ensamblaje Cooperativo de Estructuras 3D}. Zapata, H.,
Perozo, N. II Conferencia Nacional de Computación, Informática y Sistemas (CONCISA), pp. 33-40. Universidad Católica Andrés Bello, Venezuela, 2014.

\item \comillas{Middleware Reflexivo Semántico para Ambientes Inteligentes}. Mendonca, M.,
Aguilar, J., Perozo, N. II Conferencia Nacional de Computación, Informática y Sistemas (CONCISA), pp. 24-32. Universidad Católica Andrés Bello, Venezuela, 2014.

\item \textbf{Institución:} Decanato de Ciencias y Tecnología (UCLA), Barquisimeto – Venezuela.\\
\textbf{Año:} Junio, 2012.\\
\textbf{Nombre del Evento:} VII Jornada de Actualización para Análisis de Sistemas y Carreras Afines.\\
\textbf{Ponencia:} Aplicaciones de la Inteligencia Colectiva.

\item \textbf{Institución:} Decanato de Ciencias y Tecnología (UCLA), Barquisimeto – Venezuela.\\
\textbf{Año:} Octubre, 2010.\\
\textbf{Nombre del Evento:} VI Jornadas de Investigación y Postgrado.\\
\textbf{Ponencia:} Free Software Development through MASOES.

\item \textbf{Institución:} UNET, San Cristóbal – Venezuela.\\
\textbf{Año:} Octubre, 2009.\\
\textbf{Nombre del Evento:} VI Seminario Nacional de Modelos y Modelados.\\
\textbf{Ponencia:} Un Modelo Afectivo para una Arquitectura Multiagente para
Sistemas Emergentes y Auto-Organizados (MASOES).

\item \textbf{Institución:} Decanato de Ciencias y Tecnología (UCLA), Barquisimeto – Venezuela.\\
\textbf{Año:} Mayo, 2005.\\
\textbf{Nombre del Evento:} XVI Jornadas de Infociencias.\\
\textbf{Ponencia:} Aplicaciones de los Sistemas Multiagente.

\item \textbf{Institución:} Universidad de Carabobo (UC), Valencia – Venezuela.\\
\textbf{Año:} Marzo, 2005.\\
\textbf{Nombre del Evento:} II Seminario de Inteligencia Artificial: Un Panorama de
Aplicaciones.\\
\textbf{Ponencia:} Arquitectura de un Sistema Operativo Web Basado en Agentes
Inteligentes.

\item \textbf{Institución:} Universidad Fermín Toro (UFT), Barquisimeto – Venezuela.\\
\textbf{Año:} Enero, 2005.\\
\textbf{Nombre del Evento:} I Jornadas de Inteligencia Artificial.\\
\textbf{Ponencia:} Arquitectura en Sistemas Web Multiagente.

\end{enumerate}

\espaciotriple

\noindent\textbf{INTERNACIONALES}

\begin{enumerate}
\item \comillas{An Approach for Multiple Combination of Ontologies Based on the Ants Colony Optimization Algorithm}. Mendonca, M., Perozo, N., Aguilar, J. Proceedings of AsiaPacific Conference on Computer Aided System Engineering (APCASE), 2015 pp. 140-145, doi:10.1109/APCASE.2015.32.

\item \comillas{Una Ontología Emergente para Ambientes Inteligentes basada en el Algoritmo de Optimización por Colonias de Hormigas}. Mendonca, M., Perozo, N., Aguilar, J. Proceedings of the 2014 Latin American Computing Conference (CLEI), pp. 596-606. Facultad de Ingeniería, Universidad de la República, Montevideo Uruguay, 2014.

\item \comillas{The Verification Method of MASOES applied to the Social Force Model for
Pedestrian Dynamics}. Proceedings of the 9th WSEAS Int. Conf. on Computational Intelligence, Man-Machine Systems And Cybernetics (CIMMACS '10). World Scientific and Engineering Academy and Society (WSEAS) y Universidad de los Andes pp. 83-90.Venezuela, 2010.

\item \textbf{Institución:} Programme Alban – Programme de Bourses de Haut Niveau de la
l’Union Européenne pour Amérique Latine.\\
\textbf{Año:} 2009.\\
\textbf{Nombre del Evento:} 3eme Conference Alban – Porto 2009 (Portugal).\\
\textbf{Ponencia:} Experimental Study of Pedestrian Collective Displacements.

\item \textbf{Institución:} Programme Alban – Programme de Bourses de Haut Niveau de la
l’Union Européenne pour Amérique Latine.\\
\textbf{Año:} 2007.\\
\textbf{Nombre del Evento:} 2éme Conference Alban – Grenoble 2007 (Francia).\\
\textbf{Ponencia:} Esquemas de Coordinación Emergente para Plataformas
Distribuidas.

\item \textbf{Institución:} TECNOCOM 2003.\\
\textbf{Año:} Mayo 2003.\\
\textbf{Nombre del Evento:} 3era Feria y Seminario de Informática, Electrónica y Telecomunicaciones – Medellín (Colombia).\\
\textbf{Ponencia:} Propuesta de un Sistema Operativo WEB.

\item \comillas{Pattern Recognition Problems and Sparse Distributed Memory}, Proceedings of the XXVIII Latinoamerican Informatics Conference. Montevideo, Uruguay (7 Páginas, CD). Noviembre 2002.

\item \comillas{Sparse Distributed Memory with Adaptive Threshold}, Proceedings of the II Intl. Conference on Hybrid Intelligent Systems, Diciembre 2002. Santiago de Chile, Chile.
\end{enumerate}
