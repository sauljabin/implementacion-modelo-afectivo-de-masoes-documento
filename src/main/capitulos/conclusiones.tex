%%
% Copyright (c) 2017 Saúl Piña <sauljabin@gmail.com>.
%
% This program is free software: you can redistribute it and/or modify
% it under the terms of the GNU General Public License as published by
% the Free Software Foundation, either version 3 of the License, or
% (at your option) any later version.
%
% This program is distributed in the hope that it will be useful,
% but WITHOUT ANY WARRANTY; without even the implied warranty of
% MERCHANTABILITY or FITNESS FOR A PARTICULAR PURPOSE.  See the
% GNU General Public License for more details.
%
% You should have received a copy of the GNU General Public License
% along with this program.  If not, see <http://www.gnu.org/licenses/>.
%%

\capitulo{Conclusiones y Recomendaciones}

La arquitectura multiagente MASOES permite modelar un sistema emergente
y auto-organizado. Esta arquitectura
describe los elementos y relaciones a nivel individual y colectivo, que
determinan los fenómenos de emergencia y auto-organización en un sistema.
En este trabajo se abordó la implementación del modelo afectivo
propuesto en MASOES, y por ende su componente conductual,
el cual permite generar cambios dinámicos de comportamientos en los
agentes emocionales, guiados por su estado emocional.
Adicionalmente, se propone el cálculo de
la Emoción Social de un grupo de agentes,
que se compone por un conjunto de tres valores: la Emoción Central,
que determina la emoción que tienden a exhibir los agentes;
la Distancia Máxima, determina los estados emocionales más alejados del grupo;
y la Dispersión Emocional, que define la variación de estados emocionales del grupo,
es decir, dicta si los estados emocionales son homegéneos o heterogéneos.

A fin de verificar que la implementación del modelo afectivo genere correctamente emociones a nivel individual y colectivo
y que la priorización de los tipos de comportamiento sea acorde a las reglas de MASOES,
se desarrollaron casos de estudio
basados en lo modelado por \cite{perozo2012} sobre el sistema colaborativo Wikipedia.
Los resultados obtenidos demuestran que la implementación cumple
con lo especificado en MASOES, tanto a nivel individual como colectivo.
Uno de los hallazgos interesantes es la importancia que tiene la dispersión emocional
sobre la interpretación de la emoción central, esto se debe, a que
expresa cuan representativa es la emoción central frente a un grupo de agentes.
Se pudo comprobar que la emoción central es más válida a medida que la dispersión
emocional es más cercana a cero, ya que se trata de un conjunto de agentes que tienen emociones muy parecidas (homogéneas).
También, se observó que el número de agentes no influye directamente en el resultado
e interpretación de la emoción central, en otras palabras, si un grupo de agentes
es pequeño los resultados de la emoción central se comportarán de la misma forma que en
un grupo de agentes grande.
A nivel individual,
se pudo observar que los agentes generan emociones y priorizan los comportamientos
correctamente, esto se hizo a través de la comparación de los resultados
con los obtenidos a nivel de diseño \citep{perozo2012}, los cuales concuerdan inequívocamente.

Con respecto a los aportes, este trabajo de investigación no solo
provee una implementación del modelo afectivo de MASOES, sino que también,
proporciona un marco de trabajo el cual se puede seguir extendiendo,
para simular cualquier tipo de sistema emergente y auto-organizado modelado con MASOES.
Por otro lado, se construyeron diferentes utilitarios de código que permiten,
entre otras cosas, controlar la plataforma JADE, comunicar agentes, realizar pruebas funcionales y
trabajar con el lenguaje \textit{Prolog}, también, se incluyeron interfaces gráficas
que sirven para el monitoreo de agentes o para la configuración de simulaciones.
Es importante destacar que dichas interfaces pueden ser modificadas, para
adaptarse a diferentes dominios y sistemas.
La interfaz de configuración de simulación permite observar gráficos en tiempo real
acerca de la emoción social y los estados emocionales individuales de cada agente,
además, incluye la generación de resultados a través de archivos de texto.
Se desarrolló un agente que calcula la emoción social.
Este cálculo se basa en el promedio de los estados emocionales de un grupo de agentes y
la desviación estándar de dicho promedio. La emoción social es de gran importancia,
ya que describe la tendencia de los estados emocionales de un grupo de agentes,
y los agentes emocionales podrían utilizarla como herramienta para cumplir
sus objetivos colectivos.

Otro de los aportes más relevantes de este trabajo, es el diseño de una ontología de
comunicación para MASOES, específicamente para agentes estandarizados FIPA, con ella
es posible comunicar los agentes emocionales entre sí o con otros tipos de agentes,
además, es posible una comunicación entre diferentes plataformas,
es decir, los agentes escritos en lenguaje \textit{Java} pueden comunicarse
con agentes programados en lenguajes diferentes.

Otro aporte interesante, es la implementación y diseño de la Base de Conocimiento
Conductual, esta fue desarrollada con el lenguaje de programación \textit{Prolog},
ampliamente utilizado por su versatilidad y facilidad de uso, para esto
se propuso una manera estándar de definir el conocimiento asociado
al agente, a los tipos de emociones, a la priorización de comportamiento y a los estímulos,
este último de gran importancia para los agentes emocionales, ya que estos
pueden y deben registrar como afectan los estímulos a su estado emocional.
En cada estímulo se deben establecer los parámetros
que definen el incremento de la activación y satisfacción del estado emocional del agente.
A su vez, se propone un algoritmo para la
actualización del estado emocional del agente de manera incremental.

Es importante destacar, que la implementación se llevó a cabo tomando en cuenta los estándares FIPA,
lo que permite la interoperabilidad de los agentes emocionales con cualquier otro agente FIPA,
esto hace que sea estándar y fácil de extender por otros.

Como trabajo futuro, se podría implementar otros componentes individuales de la arquitectura de MASOES,
como son, los componentes Cognitivo, Reactivo y Social, y componentes colectivos como
la Base de Conocimiento Colectivo, todo esto con la finalidad de completar la implementación
de esta arquitectura y ser usada para modelar e instanciar un sistema colectivo
emergente y auto-organizado de software. Asimismo, se puede destacar que el componente conductual
presentado en este trabjao, es susceptible a modificaciones y mejoras, como por ejemplo:
se podría mejorar el diseño de clases o utilizar un lenguaje lógico diferente de \textit{Prolog}
para construir la Base de Conocimiento Conductual, o también, incluir más conocimiento
en dicha base.
Otra oportunidad de investigación, es proponer un cálculo de emoción social,
que pueda dar como resultado más de una emoción central, esto, basado en las agrupaciones de estados emocionales
que puedan emerger en el grupo de agentes, ya que como se vio, la interpretación de la emoción central propuesta
en este trabajo está sujeta al resultado de la dispersión emocional y en menor manera a la distancia máxima.

Por otra parte, sería interesante probar la implementación propuesta en este trabajo
en diferentes host, en otras palabras, que los agentes no se instancien localmente, sino
que se instancien y comuniquen en una red, lo cual sería útil en áreas como la robótica,
donde cada agente emocional sea un robot y este modifique su comportamiento
según su estado emocional, para esta propuesta se recomienda implementar previamente
al menos los componentes Cognitivo y Reactivo de MASOES.
