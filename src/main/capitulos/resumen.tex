%%
% Copyright (c) 2017 Saúl Piña <sauljabin@gmail.com>.
%
% This program is free software: you can redistribute it and/or modify
% it under the terms of the GNU General Public License as published by
% the Free Software Foundation, either version 3 of the License, or
% (at your option) any later version.
%
% This program is distributed in the hope that it will be useful,
% but WITHOUT ANY WARRANTY; without even the implied warranty of
% MERCHANTABILITY or FITNESS FOR A PARTICULAR PURPOSE.  See the
% GNU General Public License for more details.
%
% You should have received a copy of the GNU General Public License
% along with this program.  If not, see <http://www.gnu.org/licenses/>.
%%

La computación emocional es un área reciente de la inteligencia artificial que
tiene como objetivo mejorar los procesos interactivos entre agentes emocionales
y el ser humano, tanto en aplicaciones de software como de hardware. Debido a las
posibles aplicaciones en el área, actualmente la comunidad científica realiza
esfuerzos para aplicar las teorías existentes en sistemas multiagente.
Diferentes autores estudian modelos emocionales con el
objetivo de mejorar la interacción entre agentes inteligentes, un ejemplo es el modelo
afectivo de MASOES, aunque este modelo afectivo ha sido verificado formalmente
a nivel de diseño, no ha sido verificado a nivel de implementación.
Frente a lo expuesto, el presente trabajo propone una implementación del modelo afectivo de MASOES
sobre un sistema multiagente, con la finalidad de brindar un
entorno para la interacción entre los procesos emocionales y las diferentes
funciones de un agente. Adicionalmente, se propone el cálculo
de la Emoción Social, permitiendo
describir el estado emocional colectivo de un grupo de agentes emocionales. Para esto,
se diseñó y desarrolló un sistema multiagente basado en el marco de trabajo JADE,
el cual utiliza el estándar FIPA que permite el desarrollo de agentes universales.
Posteriormente, se aplicó lo implementado sobre un caso de estudio utilizando simulaciones
para generar emociones a nivel individual y colectivo, y se comparó los resultados a
nivel de implementación con los obtenidos por \cite{perozo2011} a nivel de
diseño.
