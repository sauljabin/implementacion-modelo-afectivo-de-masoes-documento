%%
% Copyright (c) 2017 Saúl Piña <sauljabin@gmail.com>.
%
% This program is free software: you can redistribute it and/or modify
% it under the terms of the GNU General Public License as published by
% the Free Software Foundation, either version 3 of the License, or
% (at your option) any later version.
%
% This program is distributed in the hope that it will be useful,
% but WITHOUT ANY WARRANTY; without even the implied warranty of
% MERCHANTABILITY or FITNESS FOR A PARTICULAR PURPOSE.  See the
% GNU General Public License for more details.
%
% You should have received a copy of the GNU General Public License
% along with this program.  If not, see <http://www.gnu.org/licenses/>.
%%

\capitulo{Marco Metodológico}

\seccion{Tipo de Investigación}

El presente estudio se enmarca dentro de la línea de investigación de
Inteligencia Artificial, más específicamente en el área de Sistemas Multiagente
y la Computación Emocional, además, se clasifica como tipo investigación de
campo, ya que el \comillas{Manual para la Elaboración del Trabajo Conducente a Grado
Académico de Especialización, Maestría y Doctorado} de la Universidad
Centroccidental Lisandro Alvarado (2002) la define como:

\begin{citatextual}
...la aplicación del método científico en el tratamiento de un sistema de
variables y sus relaciones, las cuales conducen a conclusiones y al
enriquecimiento de un campo del conocimiento o disciplina inherente a la
especialidad, con la sustentación de los experimentos y observaciones
realizadas.
\end{citatextual}

Por otra parte, la investigación estará apoyada en un diseño de tipo documental
para profundizar en los conocimientos del área de estudio, conocer el estado del
arte en este campo de investigación y aprovechar la experiencia de otros
investigadores en trabajos similares. Se usará como base investigaciones previas
como la realizada por \cite{perozo2011}, en la cual propone y describe el
modelo afectivo para MASOES. A su vez, se hará un estudio más extenso del estado
del arte a partir de la introducción de MASOES hacia el presente, lo que
permitirá contrastar los avances en el área con la propuesta original.

\seccion{Modalidad}

La investigación se establece bajo la modalidad de Proyecto Especial, el \comillas{Manual
para la Elaboración y Presentación del Trabajo Especial de Grado, Trabajo de
Grado y Tesis Doctoral del Decanato de Ciencias de la Salud} de la Universidad
Centroccidental Lisandro Alvarado (2011), lo define como:

\begin{citatextual}
...creaciones que involucran el desarrollo del ingenio y la creatividad del
investigador o investigadora. El objetivo del proyecto especial es básicamente
la creación de un producto tangible que permita solucionar problemas o
necesidades colectivas que trascienden el ámbito de las organizaciones e
instituciones. Se inscriben dentro de este tipo de investigación la producción
de software...
\end{citatextual}

En este sentido, debido a la necesidad presentada por la Unidad de Inteligencia
Artificial de la Universidad Centroccidental Lisandro Alvarado, se propone
implementar el modelo afectivo de MASOES en un sistema multiagente, como modelo
viable para la resolución de problemas asociados a modelos de agencia,
seleccionando un caso de estudio para su evaluación que contribuirá en ir
completando la implementación de MASOES en general.

\seccion{Fases del Estudio}

\subseccion{Fase I Diagnóstico}

Se comenzará haciendo una revisión bibliográfica para establecer las teorías
fundamentales que sustenten el estudio. En esta fase, se realizará el
levantamiento de información y desarrollo del estado del arte sobre los sistemas
multiagente, computación emocional y MASOES, para así poder hacer un diagnóstico
de la situación actual presentada en el área sobre estos temas e identificar
aportes recientes. Luego, se procederá a seleccionar los estudios más
relevantes, para ser descritos.

\subseccion{Fase II Diseño}

Posteriormente, se diseñará un sistema multiagente a nivel individual y
colectivo tomando como base el modelo afectivo propuesto en MASOES y el marco de
trabajo JADE. Además, se procederá a listar y analizar todas las tareas
necesarias para llevar a cabo la implementación, incluyendo los aspectos
relacionados al prototipo de software como lo serían: interfaz gráfica de
usuario, entrada y salidas de datos,  entre otros aspectos técnicos asociados al
desarrollo de software, como pruebas unitarias.

\subseccion{Fase III Implementación}

Se llevará a cabo la implementación del diseño propuesto utilizando un la
librería JADE para el desarrollo de los agentes. En esta fase se incluyen
aspectos relacionados al desarrollo de software, como programación de pruebas y
utilización de patrones de desarrollo. Además, se desarrollará la configuración
del caso de estudio a evaluar.

\subseccion{Fase IV Evaluación}

En la última fase de la investigación, se diseñarán los casos de estudio basados
en simulaciones para la generación de emociones colectivas e individuales
y se compararán los resultados obtenidos a nivel de
implementación con los resultados obtenidos por \cite{perozo2011} a nivel de
diseño.
