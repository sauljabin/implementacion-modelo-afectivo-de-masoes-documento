%%
% Copyright (c) 2017 Saúl Piña <sauljabin@gmail.com>.
%
% This program is free software: you can redistribute it and/or modify
% it under the terms of the GNU General Public License as published by
% the Free Software Foundation, either version 3 of the License, or
% (at your option) any later version.
%
% This program is distributed in the hope that it will be useful,
% but WITHOUT ANY WARRANTY; without even the implied warranty of
% MERCHANTABILITY or FITNESS FOR A PARTICULAR PURPOSE.  See the
% GNU General Public License for more details.
%
% You should have received a copy of the GNU General Public License
% along with this program.  If not, see <http://www.gnu.org/licenses/>.
%%

\introduccion

Se sabe que las emociones juegan un papel importante en el desarrollo de los
seres humanos para fines sociales o de supervivencia \citep{cuevas2015, rodriguez2015}.
Un objetivo importante planteado por la comunidad científica es
construir sistemas artificiales que exhiben comportamiento emocional, para
mejorar la interacción hombre-máquina. Los procesos emocionales se han
convertido en un requisito esencial en arquitecturas de agentes cognitivos, se
espera que el procesamiento afectivo mejore la calidad y la credibilidad de las
respuestas emocionales generadas por los agentes \citep{rodriguez2015}. Con
la aparición de la computación ubícua \citep{weiser1993} y el internet de las
cosas \citep{ashton2009} la cantidad de dispositivos heterogéneos conectados ha
aumentado considerablemente, y por ende las interacciones entre sí, y con los
humanos, haciendo que sea imperativo el desarrollo y aplicación de nuevas
técnicas que permitan una mejor respuesta de estos dispositivos, además,
tengan la capacidad de auto-organizarse, lo que traerá como consecuencia la
simplificación del diseño y la emergencia de estructuras complejas en sistemas
multiagente \citep{perozo2011}.

La computación afectiva es una de las áreas de la inteligencia artificial con
mayor interés debido a las posibles aplicaciones, puede ser usada en
simulaciones de sociedades emocionales como las del ser humano, en el
tratamiento de trastornos como el autismo, el síndrome de Asperger, la
epilepsia y depresión, así como en el reconocimiento de riesgo de estrés y su
mitigación. A su vez, se generan controversiales inferencias
sobre esta rama de la ciencia, se prevé que el impacto en el futuro de la
computación afectiva será un enorme reto para la humanidad, se plantea que es
posible que los computadores emocionales llegarán a integrarse en la sociedad
hasta el punto de necesitar derechos, al igual que los tenemos las personas,
además, la computación emocional podría reemplazar el afecto humano \citep{cuevas2015}.

En la actualidad no se conoce completamente los procesos cerebrales y mentales
asociados a las emociones, sin embargo, se realizan esfuerzos para aplicar las
teorías existentes en sistemas computacionales, diferentes autores estudian
modelos emocionales en sistemas multiagente, esto, con el objetivo de mejorar la
interacción de los agentes y ayudar a la auto-organización y emergencia en
dichos sistemas, además, incorporar emociones a agentes inteligentes
es de utilidad, debido a que las
emociones pueden hacer a los agentes más atractivos y creíbles para que puedan
desempeñar un mejor papel en diversos sistemas interactivos que involucren
simulación \citep{jiang2007}.
Un ejemplo es el modelo afectivo de MASOES \eningles{Multiagent
Architecture for Self-Organizing and Emergent Systems}, propuesto por
\cite{perozo2011}, es un modelo afectivo dimensional el cual considera un conjunto de emociones positivas y
negativas que permiten generar un cambio dinámico de comportamiento en los
agentes a nivel individual (Reactivo, Cognitivo) y colectivo (Imitativo), en
otras palabras, los agentes exhiben un comportamiento asociado a su estado
emocional actual, asimismo, los estímulos internos (Individual) o externos
(Colectivo) afectan el estado emocional de los agentes.
Esta compuesto por las dimensiones \textit{activacion} y \textit{satisfacción},
la primera representa el grado de activación fisiológica y psicológica del agente,
y la segunda el agrado o desagrado exhibido. Este modelo afectivo
para MASOES ha sido verificado \citep{perozo2011} pero no ha sido implementado,
por tal razón, se propone en este trabajo su implementación en un sistema
multiagente para evaluarlo en la generación de emociones a nivel individual y
colectivo a través de un caso de estudio que se seleccione, a fin de comparar
con los resultados ya obtenidos a nivel de la verificación del diseño.

La presente investigación se encuentra estructurada de la siguiente manera:

Capítulo I: se realiza una descripción del problema a abordar
y se plantean los objetivos de la investigación, además se incluyen la
justificación y el alcance de la investigación.

Capítulo II: comprende la revisión del estado del arte
sobre sistemas multiagente y modelos afectivos dimensionales, también se presentan las bases
teóricas que sustentan esta investigación.

Capítulo III: el marco metodológico presenta la naturaleza y las fases para llevar a cabo la
investigación.

Capítulo IV: se describe de manera detallada la propuesta planteada a nivel
de diseño e implementación.

Capítulo V: expone los casos de estudio realizados y compara los resultados con los obtenidos a nivel de diseño
por parte de \cite{perozo2011}.

Finalmente, Capítulo 6: se detallan las conclusiones y hallazgos en la presente investigación, asi
como posibles trabajos futuros.
