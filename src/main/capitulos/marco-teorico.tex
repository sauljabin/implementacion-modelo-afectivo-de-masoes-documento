%%
% Copyright (c) 2017 Saúl Piña <sauljabin@gmail.com>.
%
% This program is free software: you can redistribute it and/or modify
% it under the terms of the GNU General Public License as published by
% the Free Software Foundation, either version 3 of the License, or
% (at your option) any later version.
%
% This program is distributed in the hope that it will be useful,
% but WITHOUT ANY WARRANTY; without even the implied warranty of
% MERCHANTABILITY or FITNESS FOR A PARTICULAR PURPOSE.  See the
% GNU General Public License for more details.
%
% You should have received a copy of the GNU General Public License
% along with this program.  If not, see <http://www.gnu.org/licenses/>.
%%

\capitulo{Marco Teórico}

\seccion{Antecedentes de la Investigación}

La investigación de \cite{rodriguez2015}, es importante para el presente
trabajo ya que plantea cuatro problemas clave que tienen lugar en el desarrollo
de sistemas multiagente con computación emocional, los cuales son: \textit{la
integración de la cognición y las emociones en arquitecturas de agentes, la
unificación de los diversos aspectos de las emociones, arquitecturas escalables
para los modelos computacionales de emociones, y la explotación de la evidencia
biológica}. La integración de la cognición y las emociones en arquitecturas de
agentes, se refiere a que las arquitecturas deben proporcionar entornos
adecuados para la interacción de las funciones cognitivas y afectivas que
intervienen en los procesos emocionales, en otras palabras, la construcción
interna de los agentes debe asegurar un correcto acoplamiento entre los
diferentes componentes y el componente afectivo. La unificación de los diversos
aspectos de las emociones, teniendo en cuenta que no existe una teoría universal
que explique todo el proceso de las emociones humanas y que todos los sistemas
afectivos artificiales son desarrollados basándose en los supuestos de dichas
teorías, las arquitecturas afectivas deben proporcionar marcos adecuados para la
aplicación coherente de diferentes aspectos de las emociones humanas. El
problema de las arquitecturas escalables, viene dado a que las investigaciones y
el conocimiento de las emociones humanas se encuentran en constante cambio, las
arquitecturas afectivas deben ser diseñadas de una manera flexible para que
puedan incorporar nuevos componentes y comportamientos, y de esa manera incluir
los nuevos hallazgos. Por último, el problema de la explotación de evidencia
biológica, se refiere a que las arquitecturas deben ser diseñadas de manera que
puedan aprovechar al máximo el conocimiento generado en otras áreas científicas,
además, deben incluir una serie de componentes que imitan el funcionamiento de
las estructuras del cerebro implicadas en el procesamiento de las emociones
humanas. El presente trabajo de investigación
se enfoca en dar solución al primer problema (\textit{la
integración de la cognición y las emociones en arquitecturas de agentes}),
ya que la arquitectura individual de MASOES
describe los componentes y relaciones internas de los agentes emocionales,
para producir y priorizar un comportamiento imitativo, cognitivo o reactivo
guiado por el componente afectivo.

Sobre la arquitectura MASOES se pueden mencionar las siguientes investigaciones:

La arquitectura multiagente para sistemas emergentes y auto-organizados llamada
MASOES \eningles{Multiagent Architecture for Self-Organizing and Emergent Systems},
fue introducida por \citeauthor{perozo2011} en \citeyear{perozo2011}, como una arquitectura
multiagente para el diseño, modelado y estudio de sistemas emergentes y
auto-organizados, más recientemente ha sido utilizada entre otras cosas para
modelar el fenómeno de la auto-organización y emergencia en Wikipedia \citep{perozo2013},
desarrollo de software libre \citep{perozo2013} y sistemas
multirobot \citep{gil2015}. Esta arquitectura describe los elementos,
relaciones y mecanismos, a nivel individual y colectivo, que determinan los
fenómenos de emergencia y auto-organización en un sistema, sin modelar
matemáticamente el mismo. Uno de los aspectos más interesantes de la
arquitectura MASOES, es el hecho de considerar un conjunto de emociones
positivas y negativas generadas desde un nivel individual y colectivo, para de
esta manera promover un cambio de comportamiento dinámico en los agentes en lo
individual (Reactivo, Cognitivo) como colectivo (Imitativo). Dicha
investigación, es muy importante debido a que plantea el modelo afectivo para
MASOES, el cual será utilizado en la presente investigación como base
fundamental para la construcción de una arquitectura multiagente con agentes
emocionales.

En este sentido el trabajo realizado por \cite{gil2015}, también es clave en
el área y para este trabajo, porque demuestra la aplicabilidad de los sistemas
multiagente con modelos emocionales en un problema específico, en este caso
sistemas multirobot (a nivel de hardware). La investigación usa como base el
modelo MASOES propuesto por \cite{perozo2011}. En su investigación proponen una
arquitectura con tres niveles, el primer nivel individual proporciona las
capacidades perceptivas a cada agente y los aspectos relacionados a la conducta
y emociones del robot. El segundo nivel es colectivo, soporta los procesos de
interacción de los robots con otros individuos dentro del sistema y con el medio
ambiente. El último nivel, de los procesos de aprendizaje y gestión del
conocimiento, gestiona el conocimiento tanto individual como colectivo, así como
los procesos de aprendizaje que se producen en el sistema.

Por otra parte, también es necesario mencionar otras arquitecturas y modelos
afectivos recientes enfocados en el aspecto cognición-emoción que pueden
contribuir a la implementación y evaluación de este trabajo de investigación:

La arquitectura afectiva FATIMA \eningles{Fearnot AffecTIve Mind Architecture},
propuesta por \cite{dias2014}, es una arquitectura para la
generación de emociones en agentes de software, de manera que dichas emociones
afecten el comportamiento del mismo. En este trabajo se utiliza como primera
aproximación para la generación y evaluación de emociones el modelo afectivo OCC
\eningles{Ortony, Clore and Collins} \citep{ortony1990},
en este modelo se presentan las emociones de manera categorizada y no a través
de dimensiones como otros estudios. El proceso de evaluación de las emociones se
realiza en dos pasos, en primer lugar se determina la importancia de un evento
para el agente y se definen las variables de valoración del evento,
posteriormente, se toman las variables y se genera como resultado el estado
afectivo, el cual permite cambiar el comportamiento del agente. Cada emoción
poseerá un valor, tipo y una intensidad, el estado de ánimo del agente se puede
definir como el estado afectivo general, que está influenciado por las emociones
experimentadas por el agente, las emociones positivas aumentan el estado de
ánimo, mientras que las emociones negativas lo disminuyen. En esta investigación
no se establece una relación entre las emociones y el comportamiento, como lo
realiza el modelo afectivo para MASOES, en cambio, se muestra la relación entre
las variables de valoración  y las emociones, dicha relación es extraída de la
categorización OCC: \textit{la variable deseabilidad se relaciona con alegría y
angustia, la plausabilidad con rechazo, admiración, orgullo y vergüenza, la
capacidad de atracción con amor y odio, entre otros}. En esta arquitectura se
toma en cuenta el estado emocional que puede ser producido por agentes externos
y por el ambiente, y además se habla de un componente cultural, que está
directamente relacionado con la variable plausabilidad, puede darse el caso que
entre más positiva sean las emociones colectivas del agente, es más posible que
priorice acciones positivas para otros agentes y no  para él. Una importante
diferencia con el modelo afectivo para MASOES, es que éste promueve un
compromiso entre el comportamiento individual y colectivo en la sociedad de
agentes,  además, permite explicar aspectos de la interacción social, tales como
el grado de satisfacción, cooperación, y competencia, entre otros, mientras que
FATIMA, posee un enfoque cognitivo emocional.

\cite{dastani2014programming}, proponen extender el lenguaje de programación de agentes inteligentes llamado \comillas{2APL},
con el objetivo de integrar emociones en este.
2APL es un lenguaje de programación lógico que fue desarrollado para apoyar la programación de sistemas multiagente.
Entre otras cosas, este lenguaje le permite a los agentes recibir eventos y representar información
propia, de otros agentes o del entorno.
Posee un conjunto de tres tipos de reglas de razonamiento:
el primer tipo de reglas está diseñado para generar planes para alcanzar metas,
llamadas reglas de planificación de objetivos o reglas PG \eningles{Planning Goal};
el segundo permite procesar eventos externos, mensajes y acciones abstractas,
llevan el nombre de reglas de llamada de procedimiento o reglas de PC \eningles{Procedure Call};
el último tipo permite rehacer los planes fallidos,
llamadas reglas de reparación del plan o RP \eningles{Plan Repair}.
Con respecto a la integración de emociones, se utilizó
el modelo dimensional de emociones PAD \eningles{Pleasure, Arousal and Dominance},
específicamente la implementación llamada ALMA, para definir
el conjunto de emociones que pueden exhibir los agentes, además,
crearon una base de emociones a través de un nuevo tipo de regla para el lenguaje 2APL,
llamadas reglas de emoción o ER \eningles{Emotion Rule}, en las que se definen eventos, acciones u objetos
y la intensidad con que afectarán el estado emocional del agente.
La intensidad puede ser variada por evento,
la idea es que cada evento pueda tener su propio valor
y afecte de diferente manera a la emoción.
Esta investigación sirve de inspiración para el presente trabajo,
ya que guarda similitudes con la arquitectura individual propuesta
en MASOES, en la que se tiene una Base de Conocimiento Conductual,
un Modelo Afectivo Dimensional y se procesan eventos, acciones u objetos para generar emociones.
Además, sirve como base para justificar el uso del
lenguaje de programación lógico \textit{Prolog} en la implementación
propuesta y así definir
el conocimiento asociado a los agentes, emociones,
comportamientos, eventos, acciones u objetos que afectarán las emociones.

\cite{rincon2015}, plantea un modelo afectivo en tres dimensiones llamado SEM
\eningles{Social Emotional Model} basado en el modelo
psicológico de estados emocionales PAD \eningles{Pleasure, Arousal and Dominance},
el cual está definido por las dimensiones placer,
excitación y dominio respectivamente. En el modelo el placer representa la
medida de cuan agradable se puede sentir una emoción, la excitación mide la
intensidad de lo que se está sintiendo, y el dominio se refiere al control que
ejerce sobre el comportamiento un estado emocional. Al igual que el modelo
propuesto por \cite{perozo2012}, este modelo afectivo SEM utiliza valores
normalizados en un intervalo de [-1,1] para cada dimensión, con la diferencia
que el modelo afectivo de MASOES está fundamentado en un espacio en dos
dimensiones. Ambos modelos consideran emociones individuales y grupales, difiere
en que MASOES tiene como objetivo modelar la auto-organización y emergencia en
un sistema, utilizando el modelo afectivo para que cada agente pueda cambiar su
comportamiento dinámicamente, guiado por su estado emocional para satisfacer
dinámicamente los objetivos del sistema a través de la auto-organización de sus
actividades, mientras que SEM se centra en estudiar los estados emocionales
individuales y grupales a través del tiempo. SEM como modelo afectivo es medible
a nivel colectivo a través de una ecuación que se propone en el trabajo, en
cambio, el modelo afectivo para MASOES se centra en promover un cambio de
comportamiento dinámico e incentivar la interacción social para medir el grado
de auto-organización y emergencia alcanzado por el sistema. Esta capacidad para
caracterizar las interacciones sociales, diferencia a MASOES de otros modelos
emocionales que se centran normalmente en el estudio de la relación
cognición-emoción. De esta manera, ambos modelos son perfectamente aplicables en
simulaciones de sistemas sociales emocionales y además sobre sistemas
multiagente compuestos netamente por agentes de software.
El trabajo de \citeauthor{rincon2015}
resulta útil ya que sirve como base para la propuesta de Emoción Social descrita en esta investigación.

Otro enfoque interesante es presentado por \cite{yu2015}, que propone una
arquitectura denominada EMARL \eningles{Emotional Multiagent Reinforcement Learning},
con el objetivo de dotar a agentes inteligentes de
capacidades cognitivas y emocionales internas que pueden conducir a aprender
comportamientos cooperativos. Un aspecto relevante de este estudio es el
utilizar las emociones como un mecanismo de aprendizaje de comportamientos que
permita a los agentes la maximización de recompensas y la minimización de los
castigos. Ambas arquitecturas, MASOES y EMARL, consideran el comportamiento
colectivo como el resultado de interacciones locales de los agentes de software,
con la diferencia que MASOES utiliza las emociones para ayudar a la generación
de conocimiento colectivo de manera cooperativa, en cambio, EMARL posee un
enfoque llamado dilemas sociales, donde los agentes como entidad individual
deben decidir entre tomar acciones para obtener un beneficio propio a corto
plazo de manera egoísta o cooperar con otros agentes para obtener algún
beneficio a largo plazo. Se puede resaltar de esta arquitectura el modelo
emocional a doble capa, en la capa interna, dos funciones de derivación
emocional compiten entre sí con el fin de dominar el proceso emocional del
agente, mientras que en la capa externa una estrategia explícita de
comportamientos puede ser aprendida en base a la función de derivación emocional
ganadora. En comparación, MASOES divide su modelo afectivo en cuatro fases:
clasificación de emociones, asociación de emociones a tipos de comportamiento,
determinación de la emoción actual y determinación del tipo de comportamiento.
Un importante punto de comparación es el tipo de comportamiento que promueven
estas arquitecturas, en EMARL, se habla de comportamientos altruistas, donde los
agentes cooperan con otros, y el comportamiento egoísta donde el agente busca
satisfacer sus metas individuales, ambos comportamientos son llevados a cabo
según las metas del agente y según el beneficio que pueda obtener (dilemas
sociales), en MASOES, si se trata de una emoción positiva el agente asumirá un
comportamiento imitativo, para llevar a cabo una acción colectiva, en caso de
una emoción negativa, el agente asumirá un comportamiento reactivo o cognitivo,
para llevar a cabo una acción individual.

\seccion{Bases Teóricas}

\subseccion{Sistemas Multiagente}

La Inteligencia Artificial Distribuida (IAD) es una subárea de la Inteligencia
Artificial que ha ganado una considerable importancia debido a su capacidad de
resolver problemas complejos \citep{balaji2010}, está dividida en dos
disciplinas \citep{bond1989}, la Resolución de Problemas Distribuidos (RPD)
y los Sistemas Multiagente (SMA). La RPD considera que un problema puede ser
dividido en varios módulos, o nodos, que cooperan y comparten el conocimiento de
que disponen, quedando toda la interacción entre los nodos prefijada en tiempo
de diseño como parte integrante del sistema. Por otra parte, un SMA se puede
definir como una red de solucionadores de problemas (agentes) con un nivel muy
bajo de acoplamiento, que trabajan conjuntamente, lo que posibilita que se
enfrenten a problemas más complejos que los abordables de forma individual
\citep{perozo2011}. Los agentes autónomos y los sistemas multiagente representan una
nueva forma de analizar, diseñar e implementar sistemas de software complejos.
El enfoque basado en agentes ofrece herramientas y técnicas que tienen el
potencial de mejorar considerablemente la forma en que se conceptualizan e
implementan muchos tipos de software \citep{jennings1998}.

Según \cite{balaji2010}, el concepto más aceptado de Agente es el dado
por \cite{russell2004}, ellos definen un agente como cualquier cosa capaz
de percibir su medioambiente con la ayuda de sensores y actuar sobre él
utilizando actuadores. Por otra parte, para \cite{weiss1999}, un agente es un
sistema computacional que está situado en un ambiente, y que es capaz de tomar
acciones autónomas en ese ambiente con el fin de cumplir sus objetivos de
diseño. Para hablar de agentes inteligentes es necesario que estos estén dotados
de mecanismos de razonamiento que les permiten abordar situaciones de manera
inteligente y evolucionar por medio de la experiencia \citep{perozo2011}.

Entre las propiedades más resaltantes de los agentes inteligentes se encuentran \citep{perozo2011}:

\begin{enumerate}
\item \textbf{Autonomía:} Los agentes son autónomos en la medida en que actúan sin la
intervención humana ni de otros sistemas externos. Sin embargo, un agente
inteligente puede crear redes colaborativas con otros agentes según sus
necesidades. Esta propiedad está muy relacionada a la proactividad.
\item \textbf{Comunicación:} Los agentes tienen la capacidad de comunicarse con otros agentes
utilizando un lenguaje basado en ontologías o realizar intervenciones asíncronas
a través de comunicación indirecta.
\item \textbf{Movilidad:} Es la habilidad del agente de
moverse en el ambiente. Esta capacidad posibilita una computación menos
centralizada y más distribuida. Un agente puede alojarse en cualquier nodo de la
red y realizar sus tareas utilizando los recursos locales, para después volver a
su nodo origen llevando la información procesada.
\item \textbf{Racionalidad:} Se refiere a
que los agentes prefieren ejecutar la acción más prometedora o eficiente para
conseguir sus metas.
\item \textbf{Inteligencia:} El agente está provisto de diferentes
técnicas de inteligencia artificial, que le permiten analizar situaciones
dinámicas
\item \textbf{Razonamiento:} Es la capacidad que tiene un agente para seleccionar
comportamientos acordes a la situación actual, con la finalidad de perseguir,
detener o en su defecto abandonar un objetivo, esta propiedad está muy
relacionada con la inteligencia.
\item \textbf{Sociabilidad:} Los agentes interactúan con
otros agentes mediante algún tipo de comunicación y convenios colectivos. Esta
propiedad está muy relacionada a la cooperación, colaboración y competencia.
\end{enumerate}

Entre las características más representativas de los SMA se pueden mencionar
\citep{schweitzer2007}:

\begin{enumerate}
\item \textbf{Modularidad:} En los SMA, una distinción lógica es hecha entre los módulos y
sus interacciones. Módulos particulares (entidades, subsistemas) de un sistema
son representados por los agentes respectivos. Dependiendo de la granularidad
del modelo, cada uno de esos módulos puede estar compuesto de módulos más
pequeños. Es diferente desde un punto de vista monolítico, que trata al sistema
como un todo. Un punto de vista modular permite la reconfiguración y
extensibilidad del SMA de una manera más fácil.
\item \textbf{Redundancia:} Un SMA consiste
generalmente de un gran número de agentes, muchos de ellos similares en función
y diseño. Esto significa, por un lado, que las instancias críticas no son
representadas por un solo agente, y por otro lado, que el sistema no se cae si
un agente falla de alguna manera, brindándole robustez al sistema.
\item \textbf{Descentralización:} Un SMA no es regido por un control centralizado. En lugar de
eso, las competencias y capacidades, entre otras cosas, son distribuidas entre
los diversos agentes. Esto les permite crear un control \textit{bottom-up}, de una
manera auto-organizada, como resultado de la interacción entre los diferentes
agentes.
\item \textbf{Comportamiento Emergente:} En un SMA, la interacción entre los
agentes puede producir un comportamiento nuevo (y estable) en el nivel global de
todo el sistema. Esto representa una nueva cualidad que resulta del
comportamiento agregado de los agentes, y por lo tanto, no puede ser reducido a
los agentes individuales. Además, debido a los efectos no lineales, es
frecuentemente difícil predecir las propiedades emergentes del sistema a partir
de las propiedades individuales.
\item \textbf{Funcionalidad:} Aunque cada agente puede
tener sus propias funciones (o \textit{comportamientos}), la funcionalidad del sistema
como un todo, por ejemplo, la resolución de un problema, no es asignado a
agentes específicos sino que resulta de la interacción de los diferentes
agentes.
\item \textbf{Adaptación:} La modularidad, la descentralización y la funcionalidad
emergente son las bases para que el SMA se adapte a situaciones cambiantes. Aquí
el exceso de capacidad provista por los agentes redundantes pueden jugar un rol
importante también. Como en la evolución natural, esta asegura una reserva que
puede ser utilizada en situaciones imprevistas, es decir, para la exploración de
nuevas posibilidades u oportunidades, sin perder la funcionalidad del sistema.
La adaptación (algunas veces llamada aprendizaje colectivo) también necesita que
el sistema pueda olvidar/desaprender sus viejos estados, e interacciones, entre
otras cosas, para adaptarse a nuevas situaciones.
\end{enumerate}

Dependiendo de la manera de abordar la construcción del agente, existen algunas
arquitecturas clásicas o comunes \citep{perozo2011}:

\begin{enumerate}
\item \textbf{Arquitecturas Reactivas:} Proponen un enfoque conductista, siguiendo un modelo
estímulo-respuesta, y están formadas generalmente por agentes puramente
reactivos.
\item \textbf{Arquitecturas Deliberativas:} Contiene un modelo del mundo
simbólico y explícitamente representado. La toma de decisiones se realiza por
medio de razonamiento simbólico. Está formada por agentes basados en metas o en
la utilidad.
\item \textbf{Arquitecturas Híbridas:} Surgen a partir de numerosas
alternativas que intentan combinar lo mejor de las arquitecturas deliberativas y
reactivas.
\end{enumerate}

\subseccion{Computación Emocional}

El concepto fue introducido por \cite{picard1995}, y el objetivo de la
línea de investigación es lograr una interacción humano/computadora más
eficiente. La afectividad es una dimensión significativa del comportamiento y la
comunicación humana. Lograr que las computadoras puedan comprender nuestras
emociones y a la vez que puedan \textit{expresar} (o simular) emociones propias, sería
un paso importante para establecer un cambio cualitativo en la interactividad.
La Computación Afectiva \eningles{Affective Computing} o Computación
Emocional es una disciplina de la Inteligencia Artificial que intenta
desarrollar métodos computacionales orientados a reconocer emociones humanas y
generar emociones sintéticas \citep{causa2008}.

Según \cite{causa2008}, la computación afectiva se ocupa en resolver las
siguientes problemáticas:

\begin{enumerate}
\item El reconocimiento de emociones (y de expresiones emotivas)
humanas por parte de una computadora. El objetivo es captar aquellos signos
relacionados con la expresión de emociones y lograr interpretar estados
emocionales en función de dichos signos. Este es un tema muy complejo en el que
es difícil obtener precisión. De hecho, no existe una terminología
universalmente consensuada a la hora de referirse a estos fenómenos.
\item La simulación (o generación) de estados y expresiones emocionales con computadoras.
Se intenta que las computadoras puedan simular procesos emocionales en base a
ciertos modelos. Aquí se puede reflexionar respecto a si una computadora puede
realmente tener emociones, pero, esta disciplina sólo intenta simular dichos
procesos de forma tal que resulten verosímiles, dejando de lado estas
controversias.
\end{enumerate}

Con respecto a los agentes inteligentes, hay muchas razones para incorporar las
emociones \citep{jiang2007}. Las emociones pueden hacer a los agentes más
atractivos y creíbles para que puedan desempeñar un mejor papel en diversos
sistemas interactivos que involucren simulación. Las emociones pueden jugar un
papel funcional en el comportamiento de los seres humanos y los animales,
particularmente en sistemas sociales complejos, las emociones pueden modificar
el comportamiento físico de los agentes: un agente feliz se mueve más rápido,
mientras que un agente triste es más lento. Por otra parte, los estados
emocionales pueden afectarse según el éxito o el fracaso de metas, o de manera
inversa el estado emocional puede afectar el cumplimiento de los objetivos.
Además, las emociones pueden influir en los procesos de supervivencia del
agente, evitando situaciones riesgosas o que no cumplan con sus objetivos.

Específicamente, podemos mencionar algunos roles potenciales para las emociones
en los agentes artificiales \citep{maria2007}:

\begin{enumerate}
\item \textbf{Selección de Acciones:} Que hacer próximamente en base al estado emocional
actual.
\item \textbf{Adaptación:} Cambios en el comportamiento a corto y largo plazo debido
a los estados emocionales.
\item \textbf{Regulación Social:} Comunicación o intercambio de
información con otros vía expresiones emocionales.
\item \textbf{Integración Sensorial:} Filtrado de datos en función del estado de las
emociones y el entorno.
\item \textbf{Mecanismos de Alarma:} Reacciones, como reflejos rápidos, en situaciones críticas
que interrumpen otros procesos.
\item \textbf{Motivación:} Creando motivos como parte de un
mecanismo de imitación-emoción.
\item \textbf{Manejo de Metas:} Creación de nuevas metas o
repriorización de las existentes.
\item \textbf{Aprendizaje:} Evaluaciones emocionales en el
aprendizaje por refuerzo.
\item \textbf{Centrar la atención:} Selección de datos a procesar
basados en la evaluación emocional.
\item \textbf{Control de Memoria:} Acceso, recuperación
y disminución de elementos en memoria según estados emocionales.
\item \textbf{Procesamiento estratégico:} Selección de diferentes métodos de búsqueda basada en
el estado emocional general.
\item \textbf{Auto-modelo:} Emociones como representaciones de
cómo experimenta una situación un agente.
\end{enumerate}

\subseccion{MASOES}

La arquitectura multiagente para sistemas emergentes y auto-organizados llamada
MASOES \eningles{Multiagent Architecture for Self-Organizing and Emergent Systems},
propuesta por \cite{perozo2011}, es una herramienta para el diseño no
formal de sistemas, que produzcan un estado auto-organizado el cual emerja de
las interacciones locales entre los agentes y de los cambios que se dan en el
entorno. En esta arquitectura, cada agente puede cambiar su comportamiento
dinámicamente, guiado por su estado emocional, para satisfacer dinámicamente los
objetivos del sistema a través de la auto-organización de sus actividades.

\subsubseccion{Niveles de Interacción de MASOES}

La emergencia cognitiva colectiva es obtenida a través de tres diferentes
tipos de interacción \refpilustracion{arquitectura-multiagente-masoes}:

\begin{enumerate}
\item \textbf{Interacción Local:} Es el dinamismo e influencia (interdependencia)
estrictamente entre agentes (directa, a través de alguna forma de comunicación),
o entre agentes y el entorno (indirecta, usando un campo de acción que permite
la delimitación de un área común siguiendo un mismo conjunto de reglas).
\item \textbf{Interacción Grupal:} Es originada por el dinamismo de las interacciones locales
para favorecer la creación de redes sociales o grupos estructurados de acuerdo a
un objetivo colectivo, apoyando la gestión del conocimiento de una manera
comunitaria y colaborativa.
\item \textbf{Interacción General:} Es el resultado de la interacción de la comunidad de
agentes involucrados en el sistema conforme a los objetivos comunes.
\end{enumerate}

\begin{ilustracion}[fuente=\cite{perozo2011}, etiqueta=arquitectura-multiagente-masoes, titulo={Arquitectura de MASOES}]
\includegraphics[width=10cm]{ilustraciones/marco-teorico/arquitectura-multiagente-masoes.jpg}
\end{ilustracion}

\subsubseccion{Comportamientos de los Agentes}

\begin{enumerate}
\item Comportamiento Inconsciente o reactivo, según estímulos.
\item Comportamiento Emocional, orientado por las emociones.
\item Comportamiento Consciente, que se activan o inhiben en función de sus objetivos.
\end{enumerate}

Las emociones son usadas como un mecanismo de toma de decisiones para evaluar si
el comportamiento reactivo, cognitivo o imitativo es más conveniente o no para
una situación dada de acuerdo a los intereses individuales y colectivos.

\subsubseccion{Componentes de MASOES a Nivel Colectivo}

Los componentes a nivel colectivo son los siguientes \refpilustracion{componente-masoes}:

\begin{enumerate}
\item \textbf{Conjunto de reglas:} especifican las interacciones
entre los agentes usando solamente información local.
\item \textbf{Entorno:} es un elemento importante para las interacciones indirectas entre los agentes y
para la recolección de la información generada por la sociedad de agentes.
\item \textbf{Campo de Acción:} es delimitada por los agentes, a través de marcas
dejadas en el entorno, generalmente para coordinar sus comportamientos. En
general, existen dos tipos de coordinación entre agentes: coordinación por
comunicación directa y coordinación dentro de campos de acción (comunicación
indirecta).
\item \textbf{Base de Conocimiento Colectivo:} es la memoria social
o colectiva a la que todos los agentes tienen acceso.
\item \textbf{Retroalimentación positiva:} para promover la creación de estructuras y
cambios en el sistema.
\item \textbf{Retroalimentación negativa:} para compensar
la retroalimentación positiva y ayudar a estabilizar el patrón de comportamiento
colectivo.
\end{enumerate}

\begin{ilustracion}[fuente=\cite{perozo2011}, etiqueta=componente-masoes, titulo={Componentes de MASOES a Nivel Colectivo}]
\includegraphics[width=10cm]{ilustraciones/marco-teorico/componente-masoes.jpg}
\end{ilustracion}

\subsubseccion{Componentes de MASOES a Nivel Individual}

La arquitectura a nivel individual tiene 4 componentes: Conductual (procesos
emocionales y de cambio de comportamiento o comportamientos orientados por
emociones), Reactivo (procesos reactivos o comportamientos inconscientes),
Cognitivo (procesos deliberativos o comportamientos conscientes), y Social
(procesos sociales o comportamiento social) \refpilustracion{componentes-masoes-individual}.

\begin{enumerate}
\item \textbf{Componente Conductual:} Favorece la adaptación de cada agente con su
entorno ya que crea un modelo interno del mundo exterior que regula su
comportamiento de una manera consciente y emocional. Cada proceso de toma de
decisiones en el agente estará basado en sus objetivos individuales y
colectivos, su estado emocional, y el conocimiento adquirido de manera
individual y colectiva. Los tipos de comportamiento a considerar son imitar,
reaccionar y razonar, los cuales están enlazados a los componentes social,
reactivo y cognitivo, respectivamente. Entre los elementos que lo conforman está
el Configurador Emocional encargado de manipular las emociones del agente. En
este caso, las emociones son consideradas como señales y evaluaciones que
informan, modifican y reciben retroalimentación de los procesos reactivos,
cognitivos y sociales (de otros agentes), es en este sub-componente donde estará
el modelo afectivo. También está el Manejador de Comportamiento o Conductual,
que se encarga de activar, inhibir y priorizar algunos comportamientos en el
agente basado en el estado emocional actual, las metas del agente, su situación
social (situación de sus vecinos más cercanos), y el entorno en general. Además,
maneja todos los mecanismos responsables del cambio dinámico de comportamiento,
ya que su objetivo principal es determinar y sugerir un único tipo de
comportamiento cada vez para evitar conflictos en tiempo de ejecución. El
conocimiento asociado con la gestión de las emociones, comportamientos y
experiencias emocionales pasadas, es almacenado en la Base de Conocimiento (BC)
Conductual. El rol de las emociones es determinar el comportamiento del agente
según su estado emocional, para ello se asocian las clases de emociones a
considerar con los tipos de comportamiento que puede presentar el agente.
\item \textbf{Componente Reactivo:} Encargado de producir el comportamiento reactivo del
agente. Las reacciones son reglas asociadas a los estados emocionales ya que, se
quiere tener algunas reglas activas y otras no, de acuerdo al estado emocional
del agente y a la actividad que desarrolla en un momento determinado. Para ello
tiene un Selector de Reacciones que selecciona entre las diferentes rutinas de
comportamiento existentes, es decir, las que serán ejecutadas por el componente
reactivo de acuerdo con el estado emocional del agente. Además, posee una BC
Reactivo que es la base de conocimiento reactivo para almacenar el conjunto de
reglas gestionadas por el componente reactivo.
\item \textbf{Componente Cognitivo:} Es el responsable de producir el comportamiento
cognitivo a través de diversos mecanismos cognitivos (aprendizaje y
razonamiento), y procesos de toma de decisión (intencional o deliberativa, entre
otras). Posee un Configurador de Metas Individuales para la configuración de los
objetivos individuales y de las prioridades del agente; un Deliberador como
responsable de los mecanismos cognitivos (aprendizaje, razonamiento) y de la
toma de decisión intencional o deliberativa, entre otras; y una BC Cognitiva.
\item \textbf{Componente Social:} Debe promover conciencia en los agentes sobre el
trabajo y la experiencia de los otros agentes. Específicamente, aprovecha la
experiencia de los otros (aprendizaje social), es decir, evita el aprendizaje de
cosas que ya han aprendido sus vecinos. Este componente conecta el aprendizaje
colectivo colaborativo con el aprendizaje individual. Para ello tiene un
Configurador de Metas Colectivas para la configuración de los objetivos
colectivos y de las prioridades de los agentes. También tiene una BC Social para
almacenar, entre otras cosas, el conocimiento sobre las decisiones tomadas por
sus vecinos, es decir los agentes más cercanos. Finalmente posee un Razonador
Social para seleccionar que acción debe ser imitada y de cual agente, basado en
las metas colectivas y la utilidad obtenida en casos anteriores. La idea
principal es que cada agente pueda aprender del colectivo.
\item \textbf{Otros Elementos Generales:} Tiene un Sistema de Entrada que provee a los
agentes de información sobre el mundo donde viven. Este sistema pasa las
percepciones recibidas de manera paralela al componente reactivo, conductual,
cognitivo y social. Todos los componentes interactúan recíprocamente con esa
entrada, pero es el componente conductual el que debe establecer cual componente
tiene la prioridad más alta para responder. Posee también un conjunto de
Acciones, que son las reglas de condición-acción (si... entonces) usadas en un
proceso deliberativo (ellas reflejan el comportamiento reactivo y/o cognitivo).
Finalmente, tiene un Sistema de Salida para elegir la acción del componente
indicado por el manejador conductual, en el caso que existan varias respuestas.
\end{enumerate}

\begin{ilustracion}[fuente=\cite{perozo2011}, etiqueta=componentes-masoes-individual, titulo={Componentes de MASOES a Nivel Individual}]
\includegraphics[width=10cm]{ilustraciones/marco-teorico/componentes-masoes-individual.jpg}
\end{ilustracion}

\subsubseccion{Modelo Emocional de MASOES}

El modelo afectivo (o emocional) propuesto por \citeauthor{perozo2011} considera un conjunto de
emociones positivas y negativas generadas desde un nivel individual o colectivo,
para de esta manera promover un comportamiento individual (Reactivo, Cognitivo)
o colectivo (Imitativo) en los agentes y así, aumentar su grado de satisfacción
y por consecuencia, el nivel de auto-organización y emergencia general en el
sistema. Este modelo afectivo está representado por un espacio bidimensional,
donde el eje x representa el nivel de Activación, Excitación o Relajación del
agente (mide el grado de activación fisiológica y psicológica del agente en el
intervalo [-1, 1]), y el eje y representa el nivel de satisfacción, agrado o
desagrado, también en el intervalo [-1, 1] \refpilustracion{modelo-afectivo}.

\begin{ilustracion}[fuente=\cite{perozo2011}, etiqueta=modelo-afectivo, titulo={Modelo Afectivo para MASOES}]
\includegraphics[width=12cm]{ilustraciones/marco-teorico/modelo-afectivo.jpg}
\end{ilustracion}

El modelo afectivo está dividido en cuatro fases:

\textbf{Fase I (Configurador Emocional):} Clasificación de las emociones. En el modelo
afectivo propuesto se consideran emociones positivas y negativas generadas desde
un nivel individual o colectivo, a fin de contribuir a la generación de un
comportamiento emergente y auto-organizado en el sistema a partir de la
interacción local de los agentes. Los tipos de emociones consideradas, y el
espacio afectivo definido para MASOES, son mostrados en la \refilustracion{modelo-afectivo}. El espacio
afectivo ha sido dividido en 4 cuadrantes, donde el cuadrante I (alegría,
felicidad) y III (tristeza, depresión) representan las emociones positivas y
negativas dirigidas por la obtención de metas o logros personales (nivel
individual); y los cuadrantes II (admiración, compasión) y IV (rechazo-aversión,
ira-odio) representan las emociones positivas y negativas de tono claramente
social o interpersonal, dirigidas por las acciones de los otros agentes o
cambios en el entorno (nivel colectivo).

\textbf{Fase II (Manejador de Comportamiento):} Asociación de las emociones al tipo de
comportamiento. Para esta asociación, se le asigna a cada estado emocional del
modelo afectivo propuesto uno de los tres comportamientos considerados:
Imitativo, Cognitivo y Reactivo, de acuerdo a las reglas que se establecen \refpcuadro{comportamientos-masoes}.
Para establecer estas reglas, se considera: las emociones negativas
pueden predisponer las estrategias de resolución de problemas en los seres
humanos hacia un procesamiento local que va de lo individual a lo colectivo
(procesamiento más sistemático), mientras que las emociones positivas pueden
conducir a enfoques globales que van de lo colectivo a lo individual
(procesamiento más aproximativo).

\begin{cuadro}[etiqueta=comportamientos-masoes, titulo={Comportamientos Según el Estado Emocional del Agente}]{l|ll}
\toprule
Emoción & Tipo de Emoción & Comportamiento \\
\midrule
Felicidad & Positivo & Imitación \\
Alegría & Positivo & Imitación \\
Compasión & Positivo & Imitación \\
Admiración & Positivo & Imitación \\
Tristeza & Ligeramente Negativo & Cognitivo \\
Rechazo & Ligeramente Negativo & Cognitivo \\
Depresión & Altamente Negativo & Reactivo \\
Ira & Altamente Negativo & Reactivo \\
\bottomrule
\fuentecuadro{3}{\cite{perozo2011}}
\end{cuadro}

Por otra parte, según MASOES cada agente puede interactuar local o grupalmente.
De esta manera, si se trata de una emoción positiva el agente asumirá un
comportamiento imitativo, para llevar a cabo una acción colectiva (que va del
conocimiento colectivo al conocimiento individual) que le permita interactuar
grupalmente según los objetivos colectivos establecidos. En caso de una emoción
negativa, el agente asumirá un comportamiento reactivo o cognitivo, para llevar
a cabo una acción individual (que va del conocimiento individual al conocimiento
colectivo) que le permita interactuar localmente según los objetivos del agente
\refpilustracion{estados-comportamiento}.

\begin{ilustracion}[fuente=\cite{perozo2011}, etiqueta=estados-comportamiento, titulo={Estados Emocionales con el Tipo de Comportamiento Asociado}]
\includegraphics[width=10cm]{ilustraciones/marco-teorico/estados-comportamiento.jpg}
\end{ilustracion}

Las emociones positivas (tales como: la alegría, la felicidad, la compasión y la
admiración) conducen a un comportamiento imitativo con la idea de reproducirlo
que nos hace sentir bien a nosotros y al colectivo, mientras que las emociones
negativas (tales como: la tristeza y rechazo) nos motivan a un comportamiento
cognitivo que nos lleva a reflexionar sobre la situación actual considerando los
objetivos individuales y/o colectivos, o nos induce a un comportamiento reactivo
hacia otros en estados altamente negativo como la ira y depresión, para solo
responder de forma inmediata a la situación actual \refpcuadro{comportamientos-masoes}.
Así, para asociar los estados emocionales a un
comportamiento determinado, se tienen las siguientes reglas \refpcuadro{reglas-comportamientos-masoes}:

\newpage

\begin{cuadro}[etiqueta=reglas-comportamientos-masoes, titulo={Reglas de Priorización de Comportamientos}]{lp{7cm}}
\toprule
\textbf{Regla 1:} & Si el \textit{Estado Emocional} es \textit{Positivo} entonces priorizar \textit{Comportamiento Imitativo} \\ \hline
\textbf{Regla 2:} & Sino Si el \textit{Estado Emocional} es \textit{Ligeramente Negativo} entonces priorizar \textit{Comportamiento Cognitivo} \\  \hline
\textbf{Regla 3:} & Sino Si el \textit{Estado Emocional} es \textit{Altamente Negativo} entonces priorizar \textit{Comportamiento Reactivo} \\
\bottomrule
\fuentecuadro{2}{\cite{perozo2011}}
\end{cuadro}

\textbf{Fase III (Configurador Emocional):} Determinación de la emoción actual.

\begin{enumerate}
\item Evaluación de un evento, acción u objeto para determinar el grado de
satisfacción y activación, y luego, el estado emocional afectado. Para esta
evaluación se requiere información del mundo, tal como implicaciones de los
eventos para los agentes, los gustos o preferencias de los agentes con respecto
a objetos u otros agentes, entre otras cosas. La intensidad de la emoción
afectada viene dada por el grado de satisfacción y activación del agente, luego
de la evaluación realizada. Es necesario utilizar variables para cuantificar el
grado de satisfacción y activación del agente.
\item Modificación del actual estado emocional, si es necesario. Esta transición de
un estado a otro debe ser coherente y coordinada.
\end{enumerate}

\textbf{Fase IV (Manejador de Comportamiento):} Determinación del tipo de comportamiento.
Se modifica el comportamiento actual si es necesario, de acuerdo al estado
emocional actual y la \refcuadro{comportamientos-masoes}, como una acción resultante de la emoción
detectada en la fase anterior. De esta manera, las emociones son la expresión
dinámica y fluctuante del estado afectivo del individuo, y así, permiten cambiar
dinámicamente el tipo de comportamiento del agente de acuerdo a su situación
actual.

% \seccion{Definición de Términos Básicos}
% \hacerglosario
