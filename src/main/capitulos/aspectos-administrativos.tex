%%
% Copyright (c) 2017 Saúl Piña <sauljabin@gmail.com>.
%
% This program is free software: you can redistribute it and/or modify
% it under the terms of the GNU General Public License as published by
% the Free Software Foundation, either version 3 of the License, or
% (at your option) any later version.
%
% This program is distributed in the hope that it will be useful,
% but WITHOUT ANY WARRANTY; without even the implied warranty of
% MERCHANTABILITY or FITNESS FOR A PARTICULAR PURPOSE.  See the
% GNU General Public License for more details.
%
% You should have received a copy of the GNU General Public License
% along with this program.  If not, see <http://www.gnu.org/licenses/>.
%%

\capitulo{ASPECTOS ADMINISTRATIVOS}

\seccion{Recursos}

\begin{vinetas}
\item Lenguaje de programación Java.
\item Librería GPL JADE (“JAVA Agent DEvelopment Framework”, en inglés).
\item IDE Eclipse Mars.
\item Sistema Operativo GNU/Linux.
\item Computadora con los siguientes requerimientos mínimos: procesador Intel i5,
4GB de memoria RAM y 80GB de disco duro, monitor, ratón, teclado y tarjeta de red.
\end{vinetas}

\break

\seccion{Cronograma de Actividades}

\begin{ilustracion}[etiqueta=gantt, titulo={Cronograma de Actividades}]
	\begin{ganttchart}[
		vgrid,
		hgrid,
		bar/.style={fill=gray!70},
		newline shortcut=true,
        bar label node/.append style={align=right},
		title label font=\footnotesize,
		bar label font=\tiny]{1}{20}
		\gantttitle{Año 2016}{20} \\
		\gantttitle{Julio}{4} \gantttitle{Agosto}{4} \gantttitle{Septiembre}{4} \gantttitle{Octubre}{4} \gantttitle{Noviembre}{4}\\
		\ganttbar{Estudio de aspectos teóricos \ganttalignnewline y revisión exhaustiva \ganttalignnewline del estado del arte}{1}{2}\\
		\ganttbar{Diseño del sistema multiagente \ganttalignnewline basado en el modelo \ganttalignnewline afectivo de MASOES}{3}{4}\\
		\ganttbar{Identificación de las herramientas \ganttalignnewline computacionales a utilizar}{5}{5}\\
		\ganttbar{Implementación de la propuesta}{6}{10}\\
		\ganttbar{Identificación, descripción y \ganttalignnewline desarrollo del caso de estudio}{11}{14}\\
		\ganttbar{Evaluación de los resultados \ganttalignnewline obtenidos en caso de estudio \ganttalignnewline a nivel de implementación}{15}{18}
		\ganttlink{elem0}{elem1}
		\ganttlink{elem1}{elem2}
		\ganttlink{elem2}{elem3}
		\ganttlink{elem3}{elem4}
		\ganttlink{elem4}{elem5}
	\end{ganttchart}
\end{ilustracion}
