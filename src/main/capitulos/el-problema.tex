%%
% Copyright (c) 2017 Saúl Piña <sauljabin@gmail.com>.
%
% This program is free software: you can redistribute it and/or modify
% it under the terms of the GNU General Public License as published by
% the Free Software Foundation, either version 3 of the License, or
% (at your option) any later version.
%
% This program is distributed in the hope that it will be useful,
% but WITHOUT ANY WARRANTY; without even the implied warranty of
% MERCHANTABILITY or FITNESS FOR A PARTICULAR PURPOSE.  See the
% GNU General Public License for more details.
%
% You should have received a copy of the GNU General Public License
% along with this program.  If not, see <http://www.gnu.org/licenses/>.
%%

\capitulo{El Problema}

\seccion{Planteamiento del Problema}

La detección o simulación de emociones por parte de sistemas artificiales es un
tema de alto interés científico, especialmente por el abanico de posibles
aplicaciones prácticas, un ejemplo de esto es que los sistemas con inteligencia
artificial tengan la capacidad de percibir y tomar acciones según la emociones
que las personas presenten, predecir comportamientos colectivos basados en
emociones, o que sistemas altamente complejos y con numerosas interacciones con
otros puedan simplificar su diseño, aplicando modelos emocionales que ayuden a
generar auto-organización y emergencia \citep{perozo2011}.

El interés de incorporar emociones a agentes inteligentes, se debe a que las
emociones pueden hacer a los agentes más atractivos y creíbles para que
desempeñen un mejor papel en diversos sistemas interactivos que involucren
simulación \citep{jiang2007}. Según \cite{perozo2012}, es importante considerar
emociones en sistemas multiagente (SMA), ya que se argumenta que las emociones
permiten a los organismos adaptarse a muchas situaciones, influencian el
comportamiento a nivel individual y colectivo, intervienen en la toma de
decisiones y procesamiento de la información y tienen un efecto en la
interacción social. Basado en esto, surge la arquitectura genérica
MASOES y se recomienda su uso en simulación social multiagente, sobre todo para
casos de comparación de modelos o estudios enfocados a entender tendencias
emergentes en relación a: normas, colaboración, cooperación, tráfico peatonal o
vehicular, puesto que MASOES integra un modelo afectivo que ayuda a los
agentes a tomar decisiones según su estado de ánimo artificial.

Trabajos como el de \cite{rodriguez2015}, expresan que es necesario mucha
más investigación en el área de sistemas multiagente con modelos afectivos, y
que los principales retos para estos son: la integración de la cognición y las
emociones en arquitecturas de agentes, la unificación de los diversos aspectos
de las emociones, arquitecturas escalables para los modelos computacionales de
emociones y explotación de evidencia biológica. La integración de la cognición y
las emociones en arquitecturas de agentes, se refiere a que la construcción
interna de los agentes debe asegurar un correcto acoplamiento entre los
diferentes componentes y el componente afectivo. Con respecto a la unificación
de los diversos aspectos de las emociones, se debe proporcionar marcos adecuados
para la aplicación coherente de diferentes aspectos de las emociones humanas.
Las arquitecturas afectivas escalables deben ser diseñadas de una manera
flexible para que puedan incorporar nuevos componentes y comportamientos. Por
último, el problema de la explotación de evidencia biológica, se refiere a que
las arquitecturas deben ser diseñadas de manera que puedan aprovechar al máximo
el conocimiento generado en otras áreas científicas.

Por el escenario expuesto anteriormente, se propone aplicar el modelo afectivo
de MASOES en un sistema multiagente, en el cual se pueda observar el
comportamiento exhibido a nivel individual o colectivo de los agentes, como
consecuencia de sus estados emocionales. Aunque el modelo afectivo de MASOES ha
sido verificado formalmente a nivel de diseño \citep{perozo2011}, no ha sido
verificado a nivel de implementación, lo que ofrece la oportunidad de continuar
esta investigación. Además, dicho modelo puede ser usado en cualquier caso de
estudio y arquitectura, no solamente en MASOES, ya que fue diseñado de manera
modular. Asimismo, es interesante abordar el problema propuesto por
\cite{rodriguez2015} sobre la integración de la cognición y las emociones en
arquitecturas de agentes, donde la construcción interna de los agentes debe
asegurar un correcto acoplamiento entre los diferentes componentes y el
componente afectivo, dado que este último le proporcionará al agente la
capacidad de seleccionar un comportamiento adecuado según el estado emocional
presentado. Todo esto con la finalidad de brindar un entorno para la
interacción entre los procesos emocionales y las diferentes funciones de un
agente, e inclusive en interacciones colectivas, a su vez aportar a la
comunidad científica una implementación del reciente modelo afectivo propuesto
en MASOES.

\seccion{Objetivos}
\subseccion{Objetivo General}
Implementar el modelo afectivo de MASOES en un sistema multiagente.

\subseccion{Objetivos Específicos}

\begin{enumerate}
\item Investigar los aspectos teóricos relacionados a los sistemas multiagente y computación emocional.
\item Estudiar algunas arquitecturas multiagente con modelos afectivos incluyendo a MASOES \citep{perozo2011}.
\item Diseñar un sistema multiagente a nivel individual y colectivo basado en el modelo afectivo de MASOES y en el marco de trabajo JADE.
\item Implementar el sistema multiagente y el modelo afectivo de MASOES.
\item Aplicar la implementación en un caso de estudio seleccionado.
\item Evaluar los resultados obtenidos a nivel de implementación del modelo afectivo de MASOES con los resultados obtenidos por \cite{perozo2011} a nivel de diseño.
\end{enumerate}

\seccion{Justificación e Importancia}

La arquitectura MASOES, es una propuesta muy reciente que se ha mantenido en
continua investigación y ha sido aplicada en diferentes casos de estudio, para
el estudio del comportamiento emergente y auto-organizado de Wikipedia, el
desarrollo de software libre, y el modelo de fuerza social (tráfico peatonal).
Por otra parte, el aspecto de más interés en MASOES para esta propuesta de
investigación, es la posibilidad de considerar el estado emocional del agente
para gestionar el cambio dinámico de su comportamiento, lo que representa una
diferencia importante en el modelado de sistemas multiagente.

La propuesta de investigación es justificable debido a que se identifica a
MASOES como una interesante herramienta para el modelado de sistemas con o sin
propiedades emergentes y auto-organizadas conocidas, que permite estudiar el
comportamiento emergente y auto-organizado del sistema modelado. Así, el
implementar el modelo afectivo para MASOES permite evaluarlo a nivel de
simulación y contribuir con una propuesta genérica que puede ser usada en
cualquier sistema multiagente, no solo para MASOES.

Lo dicho anteriormente se sustenta en que las emociones juegan un papel
funcional en el comportamiento de los seres humanos y de las sociedades de
humanos, lo que hace imperativo el desarrollo de estudios en líneas de
investigación como  computación emocional y sistemas multiagente, con el
objetivo de aportar innovadoras aplicaciones de modelos emocionales para la
simulación, predicción y optimización de comportamientos basados en emociones.
Cabe señalar, que para el momento de diseñar el modelo afectivo para MASOES, era
una propuesta innovadora en la generación y gestión de emociones a nivel
colectivo, lo que resulta muy atractivo para el estudio e influencia de las
emociones en al ámbito social.

\seccion{Alcances y Limitaciones}

La propuesta de investigación se limitará a la implementación multiagente del
modelo emocional propuesto en MASOES sobre un caso de estudio, no se incorporará
un componente cognitivo complejo, debido a que se enfocará el estudio en el
componente emocional. Asimismo, este trabajo no abarca la implementación de
otros componentes de MASOES.
Además, se desarrolló una herramienta computacional donde se pueda configurar el caso de estudio y modificar
las características del entorno o configuraciones individuales y colectivas de
los agentes de manera interactiva.

Por otra parte, se usa el marco de trabajo JADE \eningles{Java Agent
DEvelopment}, para la implementación del sistema multiagente, ya que permite el
desarrollo de los agentes de acuerdo a la especificación estándar FIPA
\eningles{Foundation for Intelligent Physical Agents}, provee
un conjunto de utilitarios para el desarrollo de agentes con comportamientos complejos y
se ajusta a las necesidades del presente trabajo.
