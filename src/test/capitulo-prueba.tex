\capitulo{Capítulo de Prueba}

Este documento se generó para probar la clase \comillas{uclamsc} v\uclamscversion.

\espaciodoble\textbf{Prueba de auto cita:}

\yo

\espaciodoble\textbf{Prueba de enumeración:}

\begin{enumeracion}
	\item foo
	\item bar
	\item foobar
\end{enumeracion}

\espaciodoble\textbf{Prueba de viñetas:}

\begin{vinetas}
	\item foo
	\item bar
	\item foobar
\end{vinetas}

\espaciodoble\textbf{Prueba de enumeración en parrafo:}

Este es un parrafo de ejemplo\begin{enumeracionenparrafo}
\item primer enumerado \item segundo enumerado\end{enumeracionenparrafo}
fin de parrafo.

\break

\espaciodoble\textbf{Prueba de algoritmo y referencia a algoritmo:}

\refalgoritmo{alg}, \refpalgoritmo{alg}

\begin{algoritmo}[fuente=Prueba de fuente, titulo=Algoritmo, indice=Indice del algoritmo, etiqueta=alg]
Inicialización\;
\For{$k \leftarrow 1$ \KwTo $m$}{
    Mostrar $k$
}
\end{algoritmo}

\break

\espaciodoble\textbf{Prueba de ilustación y referencia a ilustración}

\refilustracion{ilustracion}, \refpilustracion{ilustracion}

\begin{ilustracion}[fuente=Prueba de fuente, etiqueta=ilustracion, titulo=Prueba de ilustración, indice=Indice de prueba para ilustración]
	\includegraphics[width=5cm]{escudo-ucla.jpg}
\end{ilustracion}

\break

\espaciodoble\textbf{Prueba de gráfico y referencia a gráfico}

\refgrafico{grafico}, \refpgrafico{grafico}

\begin{grafico}[fuente=Una fuente, etiqueta=grafico, titulo=Prueba de gráfico, indice=Indice de prueba para gráfico]
	\includegraphics[width=5cm]{escudo-ucla.jpg}
\end{grafico}

\break

\espaciodoble\textbf{Prueba de listado y referencia a listado:}

\reflistado{listado}, \refplistado{listado}

\letralistados[\ttfamily \fontsize{15pt}{13pt}\selectfont]

\begin{listado}[titulo=Listado prueba, indice=Indice de listado prueba, etiqueta=listado, fuente=prueba]{[LaTeX]TeX}
\begin{ilustracion}
  \includegraphics{ruta_archivo}
\end{ilustracion}
\end{listado}

\espaciodoble\textbf{Prueba literales:}

\definirliterales{
	{:=}{{$\leftarrow$}}2
	{<<}{{$\langle$}}1 {>>}{{$\rangle$}}1
}
\begin{listado}[sinleyenda, sinindice]{}
	x := <<valor>>
\end{listado}

\break

\espaciodoble\textbf{Prueba de cuadro y referencia a cuadro:}

\refcuadro{ejemplocua}, \refpcuadro{ejemplocua}

\begin{cuadro}[titulo={Piedra, Papel o Tijeras – Forma Normal}, indice=Este es el indice personalizado, etiqueta=ejemplocua]{lccc}
	\toprule
	Jugares I/II & Piedra & Papel & Tijeras\\
	\midrule
	Piedra   & (0,0) & (-1,1) & (1,-1)\\
	Papel   & (1,-1) & (0,0) & (-1,1)\\
	Tijeras   & (-1,1) & (1,-1) & (0,0)\\
	\bottomrule
	\fuentecuadro{4}{Esta es la fuente}
\end{cuadro}

\espaciodoble\textbf{Prueba de multirow:}

\begin{cuadro}[titulo={Team sheet}, indice=Ejemplo de multirow, etiqueta=ejemplomultirow]{lll}
\toprule
\multicolumn{3}{ c }{Team sheet} \\
\midrule
Goalkeeper & GK & Paul Robinson \\ \hline
\multirow{4}{*}{Defenders} & LB & Lucas Radebe \\
 & DC & Michael Duburry \\
 & DC & Dominic Matteo \\
 & RB & Didier Domi \\ \hline
\multirow{3}{*}{Midfielders} & MC & David Batty \\
 & MC & Eirik Bakke \\
 & MC & Jody Morris \\ \hline
Forward & FW & Jamie McMaster \\ \hline
\multirow{2}{*}{Strikers} & ST & Alan Smith \\
 & ST & Mark Viduka \\
 \bottomrule
\end{cuadro}

\begin{sidewaysfigure}
\espaciodoble\textbf{Prueba rotación de cuadro o imagenes:}
	\begin{cuadro}[titulo={Piedra, Papel o Tijeras – Forma Normal}, indice=Ejemplo de ratación, etiqueta=ejemplocuainvertido]{lccc}
		\toprule
		Jugares I/II & Piedra & Papel & Tijeras\\
		\midrule
		Piedra   & (0,0) & (-1,1) & (1,-1)\\
		Papel   & (1,-1) & (0,0) & (-1,1)\\
		Tijeras   & (-1,1) & (1,-1) & (0,0)\\
		\bottomrule
		\fuentecuadro{4}{Esta es la fuente}
	\end{cuadro}
\end{sidewaysfigure}

\break

\espaciodoble\textbf{Prueba de glosario:}

\agregartermino{algoritmo-evolutivo}{
	name={Algoritmos Evolutivos},
	text={algoritmo evolutivo},
	description={son algoritmos dentro de la Computación Evolutiva que se inspiran en los mecanismos biológicos de selección, cruce, reproducción, y mutación para conseguir soluciones óptimas o sub-óptimas a problemas de optimización. Están basados en la evolución de poblaciones donde cada individuo representa una solución y existe una funcion-aptitud que determina la calidad de la solución y guía el proceso evolutivo},
	plural={algoritmos evolutivos}
}

Los \glspl{algoritmo-evolutivo}

El \gls{algoritmo-evolutivo}

\hacerglosario

\break

\espaciodoble\textbf{Prueba de referencias:}

\textbf{Artículo en publicación periódicas (article):}

\verb;\cite{article}; \cite{article}

\verb;\cite{article2}; \cite{article2}

\verb;\citet{article2}; \citet{article2}

\verb;\citet*{article2}; \citet*{article2}

\verb;\citep{article2}; \citep{article2}

\verb;\citep*{article2}; \citep*{article2}

\verb;\citep[cap. 2]{article2}; \citep[cap. 2]{article2}

\verb;\citep[e.g.][]{article2}; \citep[e.g.][]{article2}

\verb;\citep[e.g.][p. 32]{article2}; \citep[e.g.][p. 32]{article2}

\verb;\citeauthor{article2}; \citeauthor{article2}

\verb;\citeauthor*{article2}; \citeauthor*{article2}

\verb;\citeyear{article2}; \citeyear{article2}

\verb;\citepyear{article2}; \citepyear{article2}

\verb;\citeptitle{article2}; \citeptitle{article2}

\espaciodoble\textbf{Libros (book):}

\verb;\cite{book}; \cite{book}

\espaciodoble\textbf{Capítulo de libro (inbook):}

\verb;\cite{inbook}; \cite{inbook}

\espaciodoble\textbf{Ponencia o presentación en evento científico (proceedings, inproceedings, conference):}

\verb;\cite{proceedings}; \cite{proceedings}

\verb;\cite{inproceedings}; \cite{inproceedings}

\verb;\cite{conference}; \cite{conference}

\espaciodoble\textbf{Boletines o Folletos (booklet):}

\verb;\cite{booklet}; \cite{booklet}

\espaciodoble\textbf{Tesis o trabajos de grado (phdthesis, mastersthesis):}

\verb;\cite{phdthesis}; \cite{phdthesis}

\verb;\cite{mastersthesis}; \cite{mastersthesis}

\espaciodoble\textbf{Referencias de fuentes audiovisuales (misc):}

\verb;\cite{misc}; \cite{misc}

\espaciodoble\textbf{Referencias de informe técnico (techreport):}

\verb;\cite{techreport}; \cite{techreport}

\espaciodoble\textbf{Referencia de fuentes electrónicas:}

\cite{causa2007computacion}

\break

\espaciodoble\textbf{Prueba de cita textual:}

Lorem ipsum dolor sit amet, consectetur adipisicing elit, sed do eiusmod tempor incididunt ut labore et dolore magna aliqua. Ut enim ad minim veniam, quis nostrud exercitation ullamco laboris nisi ut aliquip ex ea commodo consequat. Duis aute irure dolor in reprehenderit in voluptate velit esse cillum dolore eu fugiat nulla pariatur. Excepteur sint occaecat cupidatat non proident, sunt in culpa qui officia deserunt mollit anim id est laborum.

\begin{citatextual}
Lorem ipsum dolor sit amet, consectetur adipisicing elit, sed do eiusmod tempor incididunt ut labore et dolore magna aliqua. Ut enim ad minim veniam, quis nostrud exercitation ullamco laboris nisi ut aliquip ex ea commodo consequat. Duis aute irure dolor in reprehenderit in voluptate velit esse cillum dolore eu fugiat nulla pariatur. Excepteur sint occaecat cupidatat non proident, sunt in culpa qui officia deserunt mollit anim id est laborum.
(p. 10)
\end{citatextual}

\espaciodoble\textbf{Prueba de ecuación y referencia a ecuación:}

\refecuacion{testecuacion}, \refpecuacion{testecuacion}

\begin{ecuacion}{testecuacion}
	y = x + 1
\end{ecuacion}

\espaciodoble\textbf{Prueba de ecuaciones:}

\begin{ecuaciones}
	y = x + 1 \\
	z = x + 2
\end{ecuaciones}

\espaciodoble\textbf{Prueba de comillas:}

\comillas{Entre comillas}

\espaciodoble\textbf{Prueba de comando eningles:}

JADE \eningles{Java Agent Development Framework}

\espaciodoble\textbf{Prueba de código fuente (verbatim):}

\verb|int variable = 4;|

\codigoenlinea{int variable = 4;}

\espaciodoble\textbf{Prueba de espacio horizontal:}

Espacios \tab{2} horizontales

\espaciodoble\textbf{Prueba de símbolo de los reales:}

\R

\Rcuadrado

\Rcubo

\espaciodoble\textbf{Prueba de página en blanco:}
\pagenblanco

\espaciodoble\textbf{Otras referencias:}

\cite{bond2014readings}

\cite{ashton2009internet}

\cite{balaji2010introduction}

\cite{causa2007computacion}

\cite{cuevas2015emociones}

\cite{dias2014fatima}

\cite{gil2015architecture}

\cite{jennings1998roadmap}

\cite{jiang2007ebdi}

\cite{maria2007emotional}

\cite{ortony1990cognitive}

\cite{perozo2008proposal}

\cite{perozo2011modelado}

\cite{perozo2012affective}

\cite{perozo2013self}

\cite{picard1995affective}

\cite{rincon2015social}

\cite{rodriguez2015computational}

\cite{rusell2004inteligencia}

\cite{schweitzer2007brownian}

\cite{weiser1993ubiquitous}

\cite{weiss1999multiagent}

\cite{yu2015emotional}

\break

Prueba de gantt:
\begin{ilustracion}[etiqueta=gantt, titulo=Prueba de Diagrama de Gantt]
	\begin{ganttchart}[vgrid, hgrid]{1}{12}
		\gantttitle{Title}{12} \\
		\ganttgroup{Group 1}{1}{10} \\
		\ganttbar{Task 1}{1}{3} \\
		\ganttbar{Task 2}{4}{10} \\
		\ganttmilestone{Milestone 1}{11}
	\end{ganttchart}
\end{ilustracion}
